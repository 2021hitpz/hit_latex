% !Mode:: "TeX:UTF-8"
%%%%%%%%%%%%%%%%%%%%%%%%%%%%%%%%%%%%%%%%%%%%%%%%%%%%%%%%%%%%%%%%%%%%%%%%%%%%%%%%
%          ,
%      /\^/`\
%     | \/   |                CONGRATULATIONS!
%     | |    |             SPRING IS IN THE AIR!
%     \ \    /                                                _ _
%      '\\//'                                               _{ ' }_
%        ||                     hithesis v3                { `.!.` }
%        ||                                                ',_/Y\_,'
%        ||  ,                   dustincys                   {_,_}
%    |\  ||  |\          Email: yanshuoc@gmail.com             |
%    | | ||  | |            https://yanshuo.site             (\|  /)
%    | | || / /                                               \| //
%    \ \||/ /       https://github.com/dustincys/hithesis      |//
%      `\\//`   \\   \./    \\ /     //    \\./   \\   //   \\ |/ /
%     ^^^^^^^^^^^^^^^^^^^^^^^^^^^^^^^^^^^^^^^^^^^^^^^^^^^^^^^^^^^^^^
%%%%%%%%%%%%%%%%%%%%%%%%%%%%%%%%%%%%%%%%%%%%%%%%%%%%%%%%%%%%%%%%%%%%%%%%%%%%%%%%
\documentclass[fontset=fandol,type=bachelor,campus=harbin]{hithesisbook}
% 此处选项中不要有空格
%%%%%%%%%%%%%%%%%%%%%%%%%%%%%%%%%%%%%%%%%%%%%%%%%%%%%%%%%%%%%%%%%%%%%%%%%%%%%%%%
% 必填选项
% type=doctor|master|bachelor|postdoc
%%%%%%%%%%%%%%%%%%%%%%%%%%%%%%%%%%%%%%%%%%%%%%%%%%%%%%%%%%%%%%%%%%%%%%%%%%%%%%%%
% 选填选项(选填选项的缺省值已经尽可能满足了大多数需求,除非明确知道自己有什么
% 需求)
% campus=shenzhen|weihai|harbin
%   含义:校区选项,默认harbin
% glue=true|false
%   含义:由于我工规范中要求字体行距在一个闭区间内,这个选项为true表示tex自
%   动选择,为false表示区间内一个最接近版心要求行数的要求的默认值,缺省值为
%   false。
% tocfour=true|false
%   含义:是否添加第四级目录,只对本科文科个别要求四级目录有效,缺省值为
%   false
% fontset=windows|mac|ubuntu|fandol|adobe
%   含义:设置字体,若不指定会自动识别系统,然后设置字体。fandol是开源字体,自行
%   下载安装后设置使用。windows是中易字库,窝工默认常用字体,绝对没毛病。mac和
%   ubuntu 默认分别是华文和思源字库,理论上用什么字库都行。后两种字库的安装方法
%   到谷歌上百度一下什么都有了。Linux非ubuntu发行版、非x86架构机器等如何运行可到
%   github issue上讨论。
% tocblank=true|false
%   含义:目录中第一章之前,是否加一行空白。缺省值为true。
% chapterhang=true|false
%   含义:目录的章标题是否悬挂居中,规范中要求章标题少于15字,所以这个选项
%   有无没什么用,除了特殊需求。缺省值为true。
% fulltime=true|false
%   含义:是否全日制,缺省值为true。非全日制如同等学力等,要在cover中设置类
%   型,封面中不同格式
% subtitle=true|false
%   含义:论文题目是否含有副标题,缺省值为false,如果有要在cover中设置副标
%   题内容,封面中显示。
% newgeometry=one|two|no
%   含义:规范中的自相矛盾之处,版芯是否包含页眉页脚,旧方法是按照包含页眉
%   页脚来设置。该选项是多选选项,如果设置为no,则版新为旧模板的版芯设置方法,
%   如果设置该选项one或two,分别对应两种页眉页码对应版芯线的相对位置。第一种
%   是严格按照规范要求,难看。第二种微调了页眉页码位置,好一点。默认two。
% debug=true|false
%   含义:是否显示版芯框和行号,用来调试。默认否。
% openright=true|false
%   含义:博士论文是否要求章节首页必须在奇数页,此选项不在规范要求中,按个
%   人喜好自行决定。 默认否。注意,窝工的默认情况是打印版博士论文要求右翻页
%   ,电子版要求非右翻页且无空白页。如果想DIY(或身不由己DIY)在什么地方右
%   翻页,将这个选项设置为false,然后在目标位置添加`\cleardoublepage`命令即
%   可。
% library=true|false
%   含义:是否为提交到图书馆的电子版。默认否。注意:如果设置成true,那么
%   openright选项将被强制转换为false。
% capcenterlast=true|false
%   含义:图题、表题最后一行是否居中对齐(我工规范要求居中,但不要求居中对
%   齐),此选项不在规范要求中,按个人喜好自行决定。默认否。
% subcapcenterlast=true|false
%   含义:子图图题最后一行是否居中对齐(我工规范要求居中,但不要求居中对齐
%   ),此选项不在规范要求中,按个人喜好自行决定。默认否。
% absupper=true|false
%   含义:中文目录中的英文摘要在中文目录中的大小写样式歧义,在规范中要求首
%   字母大写,在work样例中是全大写。该选项控制是否全大写。默认否。
% bsmainpagenumberline=true|false
%   含义:由于本科生论文官方模板的页码和页眉格式混乱,提供这个选项自定义设
%   置是否在正文中显示页码横线,默认显示。
% bsfrontpagenumberline=true|false
%   含义:由于本科生论文官方模板的页码和页眉格式混乱,提供这个选项自定义设
%   置是否在前文中显示页码横线,默认显示。
% bsheadrule=true|false
%   含义:由于本科生论文官方模板的页码和页眉格式混乱,提供这个选项自定义设
%   置是否显示页眉横线,默认显示。
% splitbibitem=true|false
%   含义:参考文献每一个条目内能不能断页,应广大刀客要求添加。默认否。
% newtxmath=true|false
%   含义:数学字体是否使用新罗马。默认是。
% chapterbold=true|false
%   含义:本科生章标题在目录和正文中是否加粗
% engtoc=true|false
%   含义:非博士生需要添加英文目录的,手动添加,如果是博士,此开关无效
% zijv=word|regu
%   含义:字距设置为规范规定33个字还是word中34个字。默认regu。
% citetwo=comma|endash
%   含义:相邻两个参考文献中的连接符是由逗号:[1,2]还是短线[1-2]。默认endash
%%%%%%%%%%%%%%%%%%%%%%%%%%%%%%%%%%%%%%%%%%%%%%%%%%%%%%%%%%%%%%%%%%%%%%%%%%%%%%%%
\usepackage{hithesis}

\graphicspath{{figures/}}

\begin{document}
\frontmatter
% !Mode:: "TeX:UTF-8"

\hitsetup{
  %******************************
  % 注意:
  %   1. 配置里面不要出现空行
  %   2. 不需要的配置信息可以删除
  %******************************
  %
  %=====
  % 秘级
  %=====
  statesecrets={公开},
  natclassifiedindex={TM301.2},
  intclassifiedindex={62-5},
  %
  %=========
  % 中文信息
  %=========
  ctitleone={局部多孔质气体静压},%本科生封面使用
  ctitletwo={轴承关键技术的研究},%本科生封面使用
  ctitlecover={局部多孔质气体静压轴承关键技术的研究},%放在封面中使用,自由断行
  ctitle={局部多孔质气体静压轴承关键技术的研究},%放在原创性声明中使用
  csubtitle={一条副标题}, %一般情况没有,可以注释掉
  cxueke={工学},
  csubject={机械制造及其自动化},
  caffil={机电工程学院},
  cauthor={于冬梅},
  csupervisor={某某某教授},
  cassosupervisor={某某某教授}, % 副指导老师
  ccosupervisor={某某某教授}, % 联合指导老师
  % 如果是深圳本科毕业论文,需要取消注释下一行,并将内容改为“规范”中要求的封面第一页最下方的日期
  szshortcdate={2022年6月},
  % 日期自动使用当前时间,若需指定按如下方式修改:
  %cdate={超新星纪元},
  cstudentid={9527},
  cstudenttype={学术学位论文}, %非全日制教育申请学位者
  cnumber={no9527}, %编号
  cpositionname={哈铁西站}, %博士后站名称
  cfinishdate={20XX年X月---20XX年X月}, %到站日期
  csubmitdate={20XX年X月}, %出站日期
  cstartdate={3050年9月10日}, %到站日期
  cenddate={3090年10月10日}, %出站日期
  %(同等学力人员)、(工程硕士)、(工商管理硕士)、
  %(高级管理人员工商管理硕士)、(公共管理硕士)、(中职教师)、(高校教师)等
  %
  %
  %=========
  % 英文信息
  %=========
  etitle={Research on key technologies of partial porous externally pressurized gas bearing},
  esubtitle={This is the sub title},
  exueke={Engineering},
  esubject={Computer Science and Technology},
  eaffil={\emultiline[t]{School of Mechatronics Engineering \\ Mechatronics Engineering}},
  eauthor={Yu Dongmei},
  esupervisor={Professor XXX},
  eassosupervisor={XXX},
  % 日期自动生成,若需指定按如下方式修改:
  edate={December, 2017},
  estudenttype={Master of Art},
  %
  % 关键词用“英文逗号”分割
  ckeywords={\TeX, \LaTeX, CJK, 嗨!, thesis},
  ekeywords={\TeX, \LaTeX, CJK, template, thesis},
}

\begin{cabstract}

摘要的字数(以汉字计),硕士学位论文一般为500 $\sim$ 1000字,博士学位论文为1000 $\sim$ 2000字,
均以能将规定内容阐述清楚为原则,文字要精练,段落衔接要流畅。摘要页不需写出论文题目。
英文摘要与中文摘要的内容应完全一致,在语法、用词上应准确无误,语言简练通顺。
留学生的英文版博士学位论文中应有不少于3000字的“详细中文摘要”。

  关键词是为了文献标引工作、用以表示全文主要内容信息的单词或术语。关键词不超过 5
  个,每个关键词中间用分号分隔。(模板作者注:关键词分隔符不用考虑,模板会自动处
  理。英文关键词同理。)
\end{cabstract}

\begin{eabstract}
   An abstract of a dissertation is a summary and extraction of research work
   and contributions. Included in an abstract should be description of research
   topic and research objective, brief introduction to methodology and research
   process, and summarization of conclusion and contributions of the
   research. An abstract should be characterized by independence and clarity and
   carry identical information with the dissertation. It should be such that the
   general idea and major contributions of the dissertation are conveyed without
   reading the dissertation.

   An abstract should be concise and to the point. It is a misunderstanding to
   make an abstract an outline of the dissertation and words ``the first
   chapter'', ``the second chapter'' and the like should be avoided in the
   abstract.

   Key words are terms used in a dissertation for indexing, reflecting core
   information of the dissertation. An abstract may contain a maximum of 5 key
   words, with semi-colons used in between to separate one another.
\end{eabstract}
 % 封面
\makecover
\begin{denotation}
\begin{table}[h]%此处最好是h
\caption{国际单位制中具有专门名称的导出单位}
\vspace{0.5em}\centering\wuhao
\begin{tabular}{ccccc}
\toprule
量的名称&单位名称&单位符号&其它表示实例\\
\midrule
频率&赫[兹]&Hz&s-1\\
\bottomrule
\end{tabular}
\end{table}
\end{denotation}
%物理量名称表,符合规范为主,有要求添加
\tableofcontents %目录
\mainmatter
% !Mode:: "TeX:UTF-8"

\chapter[哈尔滨工业大学研究生学位论文撰写规范]{哈尔滨工业大学研究生学
  位论文\protect\\撰写规范}[Harbin Institute of Technology Postgraduate Dissertation Writing Specifications]

研究生学位论文是研究生科学研究工作的全面总结,是描述其研究成果、代表其研究水平的
重要学术文献资料,是申请和授予相应学位的基本依据。学位论文撰写是研究生培养过程的
基本训练之一,必须按照确定的规范认真执行。研究生应严肃认真地撰写学位论文,指导教
师应加强指导,严格把关。

学位论文撰写应实事求是,杜绝造假和抄袭等行为;应符合国家及各专业部门制定的有关标
准,符合汉语语法规范。

硕士和博士学位论文,除在字数、理论研究的深度及创造性成果等
方面的要求不同外,撰写规范要求基本一致。人文与社会科学、管理学科可在本撰写规范的
基础上补充制定专业的学术规范。

\section{内容要求}[Content specification]

\subsection{题目}[Title]

题目应以简明的词语,恰当、准确、科学地反映论文最重要的特定内容(一般不超过25字),
应中英文对照。题目通常由名词性短语构成,不能含有标点符号;应尽量避免使用不常用的
缩略词、首字母缩写字、字符、代号和公式等。

如题目内容层次很多,难以简化时,可采用题目和副题目相结合的方法。
题目与副题目字数之和不应超过35字,中文的题目与副题目之间用破折号相连,英文则用冒
号相连,\emph{除此之外不能含有标点符号}。
副题目起补充、阐明题目的作用。题目和副题目在整篇学位论文中的不同地方出现时,应保持一致。

\subsection{摘要与关键词}[Abstraction and key words]
\subsubsection{摘要}[Abstraction]

摘要是论文内容的高度概括,应具有独立性和自含性,即不阅读论文的全文,就能通过摘要
了解整个论文的必要信息。摘要应包括本论文研究的目的、理论与实际意义、主要研究内容、
研究方法等,重点突出研究成果和结论。

摘要的内容要完整、客观、准确,应做到不遗漏、不拔高、不添加。摘要应按层次逐段简要
写出,避免将摘要写成目录式的内容介绍。摘要在叙述研究内容、研究方法和主要结论时,
除作者的价值和经验判断可以使用第一人称外,一般使用第三人称,采用“分析了……原因”、
“认为……”、“对……进行了探讨”等记述方法进行描述。避免主观性的评价意见,避免
对背景、目的、意义、概念和一般性(常识性)理论叙述过多。

摘要需采用规范的名词术语(包括地名、机构名和人名)。对个别新术语或无中文译文的术
语,可用外文或在中文译文后加括号注明外文。摘要中不宜使用公式、化学结构式、图表、
非常用的缩写词和非公知公用的符号与术语,不标注引用文献编号。

博士学位论文摘要应包括以下几个方面的内容:

(1)论文的研究背景及目的。简洁准确地交代论文的研究背景与意义、相关领域的研究现
状、论文所针对的关键科学问题,使读者把握论文选题的必要性和重要性。此部分介绍不宜
写得过多,一般不多于400字。

(2)论文的主要研究内容。介绍论文所要解决核心问题开展的主要研究工作以及研究方法
或研究手段,使读者可以了解论文的研究思路、研究方案、研究方法或手段的合理性与先进
性。

(3)论文的主要创新成果。简要阐述论文的新思想、新观点、新技术、新方法、新结论等
主要信息,使读者可以了解论文的创新性。

(4)论文成果的理论和实际意义。客观、简要地介绍论文成果的理论和实际意义,使读者
可以快速获得论文的学术价值。

\subsubsection{关键词}[Keywords]
关键词是供检索用的主题词条。关键词应集中体现论文特色,反映研究成果的内涵,具有语
义性,在论文中有明确的出处,并应尽量采用《汉语主题词表》或各专业主题词表提供的规
范词,应列取3$\sim$6个关键词,按词条的外延层次从大到小排列。

\subsection{目录}[Content]

论文中各章节的顺序排列表,包括论文中全部章、节、条三级标题及其页码。

\subsection{论文正文}[Main body]

论文正文包括绪论、论文主体及结论等部分。

\subsubsection{绪论}[Introduction]
绪论一般作为第1章。绪论应包括:本研究课题的来源、背景及其理论意义与实际意义;国
内外与课题相关研究领域的研究进展及成果、存在的不足或有待深入研究的问题,归纳出将
要开展研究的理论分析框架、研究内容、研究程序和方法。

绪论部分要注意对论文所引用国内外文献的准确标注。绪论的主要研究内容的撰写宜使用将
来时态,切忌将论文目录直接作为研究内容。

\subsubsection{论文主体}[Main body]
论文主体是学位论文的主要部分,应该结构严谨,层次清晰,重点突出,文字简练、通顺。
论文各章之间应该前后关联,构成一个有机的整体。论文给出的数据必须真实可靠,推理正
确,结论明确,无概念性和科学性错误。对于科学实验、计算机仿真的条件、实验过程、仿
真过程等需加以叙述,避免直接给出结果、曲线和结论。引用他人研究成果或采用他人成说
时,应注明出处,不得将其与本人提出的理论分析混淆在一起。

论文主体各章后应有一节“本章小结”,实验方法或材料等章节可不写“本章小结”。各章
小结是对各章研究内容、方法与成果的简洁准确的总结与概括,也是论文最后结论的依据。

\subsubsection{结论}[Conclusion]
结论作为学位论文正文的组成部分,单独排写,不加章标题序号,不标注引用文献。
结论内容一般在\num{2000}字以内。
结论应是作者在学位论文研究过程中所取得的创新性成果的概要总结,不能与摘要混为一谈。
博士学位论文结论应包括论文的主要结果、创新点、展望三部分,在结论中应概括论文的核
心观点,明确、客观地指出本研究内容的创新性成果(含新见解、新观点、方法创新、技术
创新、理论创新),并指出今后进一步在本研究方向进行研究工作的展望与设想。
对所取得的创新性成果应注意从定性和定量两方面给出科学、准确的评价,分(1)、(2)、
(3)…条列出,宜用“提出了”“建立了”等词叙述。
此外,结论的撰写还应符合以下基本要求:
(1)结论具有相对的独立性,不应是对论文中各章小结的简单重复。
结论要与引言相呼应,以自身的条理性、明确性、客观性反映论文价值。对论文创新内容的概括、评价要适当。
(2)结论措辞要准确、严谨,不能模棱两可,避免使用“大概”“或许”“可能是”等词
语。
结论中不应有解释性词语,而应直接给出结果。结论中一般不使用量的符号,而宜用量的名称。
(3)结论应指出论文研究工作的局限性或遗留问题,如条件所限,或存在例外情况,或本论文尚难以解释或解决的问题。
(4)常识性的结果或重复他人的结果不应作为结论。

\subsection{参考文献}[Reference]
所有被引用文献均要列入参考文献中,必须按顺序标注,但同一篇文章只用一个序号。
尽量引用原始文献。当不能引用原始文献时,要将二次引用文献、原始文献同时标注。
博士学位论文的参考文献一般不少于100篇,硕士学位论文的参考文献一般不少于40篇,其
中外文文献一般不少于总数的1/2。参考文献中近五年的文献数一般应不少于总数的1/3,并
应有近两年的参考文献。
教材、产品说明书、国家标准、未公开发表的研究报告(著名的内部报告如PB、AD报告及著
名大公司的企业技术报告等除外)等通常不宜作为参考文献引用。

引用网上参考文献时,应注明该文献的准确网页地址,网上参考文献和各类标准不包含在上
述规定的文献数量之内。
本人在攻读学位期间发表的学术论文不应列入参考文献中。

\subsection{攻读学位期间取得创新性成果}[Publications]
学位论文后应列出研究生在攻读学位期间发表的与学位论文内容相关的学术论文(含已录用
的论文)。
攻读学位期间所获得的科研成果、专利可单做一项分别列出。
与学位论文无关的学术论文、署名为第二作者(不含第一作者为导师和副导师)以后的学术
论文,不宜在此列出。

\subsection{原创性声明及使用权限}[Authorization]
作者可直接下载本部分内容电子版。作者和导师本人签署姓名。

\subsection{致谢}[Acknowledgments]
对导师和给予指导或协助完成学位论文工作的组织和个人,对课题给予资助者表示感谢。内容应简朴、语言应含蓄。

\subsection{个人简历}[Resume]
包括学习经历和工作经历。

\section{书写规定}[Regulation]
\subsection{论文正文字数}[Word number]
博士学位论文正文一般为6万$\sim$10万字(含图表)。
硕士学位论文正文一般为3万$\sim$5万字(含图表)。
\subsection{论文书写}[Writing]
研究生学位论文一律要求在计算机上输入、编排与打印。
页码在版心下边线之下居中排放;摘要、目录、物理量名称及符号表等文前部分的页码用罗马数字单独编排,正文以后的页码用阿拉伯数字编排。
硕士学位论文的扉页、摘要,博士学位论文的扉页、摘要、目录、图题及表题等,都要求用中、英文两种文字给出,具体编排上中文在前。
留学生和外语专业的学位论文的扉页、摘要及目录,要求用中、英文两种文字给出,其他用英文或所学专业相应的语言撰写。扉页、摘要及目录的英文部分另起一页。

\subsection{摘要与关键词}[Abstract]
摘要的字数(以汉字计),硕士学位论文一般为500$\sim$\num{1000}字,博士学位论文为
\num{1000}$\sim$\num{2000}字,均以能将规定内容阐述清楚为原则,文字要精练,段落衔
接要流畅。
摘要页不需写出论文题目。
英文摘要与中文摘要的内容应完全一致,在语法、用词上应准确无误,语言简练通顺。
留学生的英文版学位论文中应有不少于\num{3000}字的“详细中文摘要”。
关键词在摘要后列出,中英文关键词应一一对应,分别排在中英文摘要下方,中文关键词之间用“;”隔开,英文关键词之间用“,”隔开。

\subsection{目录}[Contents]
目录应包括论文中全部章、节、条三级标题及其页码,含:
摘要
Abstract
物理量名称及符号表(参照附录1,采用国家标准规定符号者可略去此表)
正文章节题目(要求编到第3级标题,即×.×.×)
结论
参考文献
附录
攻读□士学位期间发表的学术论文(□为“博”或“硕”)
原创性声明
使用授权说明
索引(可选择或不选择)
致谢
个人简历

\subsection{论文正文}[Main text]
\subsubsection{章节及各章标题}[Titles]
论文正文分章节撰写,每章应另起一页。
各章节标题要突出重点、简明扼要。
字数一般应在15字以内,标题中不加标点符号。
标题中尽量不采用英文缩写词,必须采用时应使用本行业的通用缩写词。
\subsubsection{层次}[Hierarchy]
层次以少为宜,应根据实际需要选择。
层次代号建议采用本文3.7中表1的格式。
层次要求统一,若节下内容无须列条,可直接列项。具体用到哪一层次,视需要而定。
\subsection{引用文献标注}[Reference]
引用文献标注遵照《信息与文献参考文献著录规则》(GB/T7714—2015),采用顺序编码制。
正文中引用文献的标示应置于所引内容最后一个字的右上角,所引文献编号用阿拉伯数字置
于方括号“[ ]”中,用小4号字体的上角标。
要求:
(1)引用单篇文献时,如“二次铣削\cite{cnproceed}”。

(2)同一处引用多篇文献时,各篇文献的序号在方括号内全部列出,各序号间用“,”,如
遇连续序号,可标注讫序号。如,…形成了多种数学模型\cite{cnarticle,cnproceed}…
注意此处添加\cs{inlinecite}中文空格\inlinecite{cnarticle,cnproceed},可以在cfg文件中修改空格类型。

(3)多次引用同一文献时,在文献序号的“[ ]”后标注引文页码。如,…间质细胞CAMP含量
测定\cite[100-197]{cnarticle}…。…含量测定方法规定
\cite[92]{cnarticle}…。

(4)当提及的参考文献为文中直接说明时,则用小4号字与正文排齐,如“由文献\inlinecite{hithesis2017}可知”

\subsection{名词术语}[Glossary]
科技名词术语及设备、元件的名称,应采用国家标准或部颁标准中规定的术语或名称。
标准中未规定的术语要采用行业通用术语或名称。
全文名词术语必须统一,一些特殊名词或新名词应在适当位置加以说明或注解。
采用英语缩写词时,除本行业广泛应用的通用缩写词外,文中第一次出现的缩写词应该用括号注明英文原词。

\subsection{物理量标注}[Symbols]
\subsubsection{物理量的名称和符号}[Name and symbols]
物理量的名称和符号应符合《国际单位制及其应用》(GB 3100—93)、《量和单位》(GB
3102.1$\sim$13—93)的规定。
论文中某一物理量的名称和符号应统一。
物理量的符号必须采用斜体。

\subsubsection{物理量计量单位}[Units]
物理量计量单位及符号应按国务院1984年发布的《中华人民共和国法定计量单位》(见附录
1)及《国际单位制及其应用》(GB 3100—93)、《量和单位》(GB 3102.1$\sim$13—93)
执行,不得使用非法定计量单位及符号。
计量单位可采用汉字或符号,但应前后统一。计量单位符号,除用人名命名的单位第一个字母用大写之外,一律用小写字母。
非物理量单位(如件、台、人、元、次等)可以采用汉字与单位符号混写的方式,如“万t·km”、“t/(人·a)”等。
不定数字之后可用中文计量单位符号,如“几千克”。
表达时刻时应采用中文计量单位,如“上午8点3刻”,不能写成“8h45min”。
计量单位符号一律用正体。

\subsection{外文字母的正体与斜体用法}[English]
按照《国际单位制及其应用》(GB 3100—93)、《量和单位》(GB 3102.1$\sim$13—93)的
规定,物理量符号、物理常量、变量符号用斜体,计量单位等符号用正体。外文字母采用
Times New Roman(新罗马)字体。

\subsection{数字}[Number]
按《出版物上数字用法》(GB/T 15835—2011),除习惯用中文数字表示的以外,一般均采
用阿拉伯数字(参照附录2),Times New Roman字体。

\subsection{公式}[Equation]
论文中的公式应另起行,并居中书写,与周围文字留有足够的位置区分开。公式应标注序号,
并将序号置于括号内。公式序号按章编排,如第1章第1个公式的序号为“(1-1)”。公式
的序号右端对齐。
文中引用公式时,一般用“见式(1-1)”或“由公式(1-1)”。
若公式前有文字(如“解”“假定”等),文字前空4个半角字符,公式仍居中排,公式末不加标点。
公式中用斜线表示“除”的关系时应采用括号,以免含糊不清,如。通常“乘”的关系在前,如,而不写成。
公式较长时最好在“=”(等号)处转行,如难实现,则可在“+、-、×、÷”运算符号
处转行,转行时运算符号仅书写于转行式前,不重复书写。
公式中第一次出现的物理量代号应给予注释,注释的转行应与破折号“——”后第一个字对齐。
破折号占4个半角字符,注释物理量需用公式表示时,公式后不应出现公式序号,如(3-1)。
公式中应注意分数线的长短(主、副分数线严格区分),长分数线与等号对齐,不能用文字
形式表示等式。
公式中变量下标按《量和单位》中规定,建议用正体形式。

\begin{equation}\label{eq:2}
  \ddot{\boldsymbol{\rho}}-\frac{\mu}{R_{t}^{3}}\left(3\mathbf{R_{t}}\frac{\mathbf{R_{t}\rho}}{R_{t}^{2}}-\boldsymbol{\rho}\right)=\mathbf{a}
\end{equation}
\begin{tabularx}{\textwidth}{@{}l@{\quad}r@{———}X@{}}
  式中& $\boldsymbol{\rho}$ &追踪飞行器与目标飞行器之间的相对位置矢量;\\
      &  $\boldsymbol{\ddot{\rho}}$&追踪飞行器与目标飞行器之间的相对加速度;\\
      &  $\mathbf{a}$   &推力所产生的加速度;\\
      &  $\mathbf{R_t}$ & 目标飞行器在惯性坐标系中的位置矢量;\\
      &  $\omega_{t}$ & 目标飞行器的轨道角速度;\\
      &  $\mathbf{g}$ & 重力加速度,$\mathbf{g}=\frac{\mu}{R_{t}^{3}}\left(
                        3\mathbf{R_{t}}\frac{\mathbf{R_{t}\rho}}{R_{t}^{2}}-\boldsymbol{\rho}\right)=\omega_{t}^{2}\frac{R_{t}}{p}\left(
                        3\mathbf{R_{t}}\frac{\mathbf{R_{t}\rho}}{R_{t}^{2}}-\boldsymbol{\rho}\right)$,这里~$p$~是目标飞行器的轨道半通径。
\end{tabularx}\vspace{\ccwd}

\subsection{插表}[Table]
表应有自明性。表格不加左、右边线。表的编排建议采用国际通行的三线表,如果三线表不足以清晰表达表中内容,应加大栏与栏间距,以清晰明了为主,例如附录1中的表2。表中文字用宋体、Times New Roman字体,字号尽量采用5号字(当字数较多时可用小5号字,但在一个插表内字号要统一)。
每个表格均应有表题(由表序和表名组成)。表序一般按章编排,如第1章第一个插表的序号为“表1-1”。表序与表名之间空2个半角字符,表名中不允许使用标点符号,表名后不加标点。表题置于表上,用中、英两种文字居中排写,中文在上,用宋体5号字,英文用Times New Roman字体5号字。硕士学位论文只用中文表题。
表头设计应简单明了,尽量不用斜线。表头中可采用化学符号或物理量符号。
全表如用同一单位,则将单位符号移至表头右上角,加圆括号。
表中数据应准确无误,书写清楚。数字空缺的格内加横线“—”(占2个半角字符)。表内文字或数字上、下或左、右相同时,采用通栏处理方式,不允许用“〃”“同上”之类的写法。
表内文字说明,起行空2个半角字符,转行顶格,句末不加标点。
插表之前文中必须有相关文字提示,如“见表1-1”“如表1-1所示”。一般情况下插表不能拆开两页编排,如某表在一页内安排不下时,才可转页,以续表形式接排。表右上角注明编号,编号后加“(续表)”,并重复表头。插表的上下与文中文字间需空一行编排。
引用文献中的表格时,除在正文文字中标注参考文献序号以外,还必须在表题的右上角标注参考文献序号。
2.13  插图
图应有自明性。插图应与文字紧密配合,文图相符,内容正确。选图要力求精练,插图、照片应完整清晰。

机械工程图:采用第一角投影法,严格按照《技术制图图样画法指引线和基准线的基本规定》
(GB/T 4457.2—2003)、《机械制图机构运动简图用图形符号》(GB/T 4460—2013)、《技
术制图简化表示法》(GB/T 16675.1$\sim$2—2012)、《产品几何技术规范(GPS)技术产
品文件中表面结构的表示法》(GB/T 131—2006)及《机械工程CAD制图规则》(GB/T
14665—2012)有关规定。

数据流程图、程序流程图、系统流程图等按《信息处理 数据流程图、程序流程图、系统流
程图、程序网络图和系统资源图的文件编制符号及约定》(GB/T 1526—1989)规定。

电气图:图形符号、文字符号等应符合附录3所列有关标准的规定。

流程图:必须采用结构化程序并正确运用流程框图。

对无规定符号的图形应采用该行业的常用画法。
坐标图的坐标线均用细实线,粗细不得超过图中曲线;有数字标注的坐标图,必须注明坐标单位。
照片图要求主题和主要显示部分的轮廓鲜明,便于制版。如用放大或缩小的复制品,必须清
晰,反差适中。照片上应有表示目的物尺寸的标度。
引用文献中的图时,除在正文文字中标注参考文献序号以外,还必须在图题的右上角标注参考文献序号。

\subsubsection{图题及图中说明}[Legend]
每个图均应有图题(由图序和图名组成),图题不宜有标点符号,图名在图序之后空2个半角字符排写。图序按章编排,如第1章第一个插图的图号为“图1-1”。图题置于图下,用中、英两种文字,居中书写,中文在上,要求中文用宋体5号字,英文用Times New Roman字体5号字。有图注或其他说明时应置于图题之上。引用图应注明出处,在图题右上角加引用文献号。图中若有分图时,分图题置于分图之下或图题之下,可以只用中文书写,分图号用(a)、(b)等表示,在图题之下连续排列,用“;”间隔。
图中各部分说明应采用中文(引用的外文图除外)或数字符号,各项文字说明置于图题之上(有分图时,置于分图题之上)。
图中文字用宋体、Times New Roman字体,字号尽量采用5号字(当字数较多时可用小5号字,以清晰表达为原则,但在一个插图内字号要统一)。同一图内使用文字应统一。图表中物理量、符号用斜体。

\subsubsection{插图编排}[Figures]
插图之前,文中必须有关于本插图的提示,如“见图1-1”“如图1-1所示”等。插图与其图题为一个整体,不得拆开排写于两页。插图处的该页空白不够排写该图整体时,则可将其后文字部分提前排写,将图移到次页。有分图时,分图过多在一页内安排不下时,可转到下页,总图题只出现在该页,下页标“图序(续图)”字样。
插图的上下与文中文字间需留一定位置编排。

\subsection{参考文献}[Reference]
参考文献标注采用顺序编码制,著录格式应遵照《信息与文献  参考文献著录规则》(GB/T 7714—2015)的要求。参考文献及电子文献载体标志代码、著录细则、参考文献著录格式见附录4。以下是论文中常用的六种参考文献类型标注形式。

(1)图书文献。

[1]唐绪军. 报业经济与报业经营[M]. 北京:新华出版社,1999:117-121.

[2]霍斯尼 R K. 谷物科学与工艺学原理[M]. 李庆龙,译. 北京:中国仪器出版社,1989:32-35.

(2)期刊论文。

[1]覃睿,田先钰. 从创新潜力到创新成果:一个创新潜力形成与释放模型[J]. 科技进步与对策,2007(2):148-152.

(3)学术会议。

[1]张佐光,张晓宏,仲伟虹,等. 多相混杂纤维复合材料拉伸行为分析[C]//第九届全国复合材料学术会议论文集(下册). 北京:世界图书出版公司,1996:410-416.

(4)学位论文。

[1]金宏. 导航系统的精度及容错性能的研究[D]. 北京:北京航空航天大学,1998:60-63.

(5)电子文献。

[1] 数字化转型 2.0 时代,未来的人才与组织要如何定义?[EB/OL]( 2019-10-24) [2020-01-02]. http: // www.chinatradenews. com.cn / content /201910 /24 / c87965.html.

(6)报告。

[1]  中国互联网信息中心.第45次中国互联网络发展情况统计报告[R]. 中华人民 共和国国家互联网信息办公室,2020:1.

\subsection{附录}[Appendix]
附录作为主体部分的补充,并不是必需的。
下列内容可以作为附录置于论文后:
(1)为了整篇论文材料的完整,但编入正文又有损于编排的条理性和逻辑性,这一材料包括比正文更为详尽的信息、研究方法和技术更深入的叙述,对了解正文内容有用的补充信息等。
(2)由于篇幅过大或取材于复制品而不便于编入正文的材料。
(3)不便于编入正文的罕见珍贵资料。
(4)对一般读者并非必要阅读,但对本专业同行有参考价值的资料。
(5)某些重要的原始数据、数学推导、结构图、统计表、自编的计算机程序、计算机打印输出件等。

\subsection{攻读学位期间取得创新性成果}[Publications]
书写格式与参考文献相同,页码后需注明该文章对应学位论文的章节序号。如已发表的学术论文被EI或SCI收录,应标明收录号;SCI论文一般应标注发表当年的影响因子;对已录用但尚未发表的学术论文,请注明是否EI或SCI刊源。

\subsection{索引}[Index]
为便于检索文中内容,可编制索引置于论文之后(根据需要决定是否设置)。索引以论文中的专业词语为检索线索,指出其相关内容的所在页码。索引用中、英两种文字书写,中文在前。中文按各词汉语拼音第一个字母排序,英文按该词第一个英文字母排序。索引示例见附录5。
\subsection{个人简历}[Resume]
除全日制硕士生外,其余学生均增列此项。个人简历一般应包含大学起的学习经历和工作经历。

\subsection{书脊}[Ridge]
为了便于学位论文的管理,建议参照《图书和其它出版物的书脊规则》(GB/T 11668—1989)规定,在学位论文书脊中标注学位论文题目及学位授予单位名称,用小4号黑体。

\subsection{其他}[Else]
年代前必须注明世纪,如20世纪70年代。


% Local Variables:
% TeX-master: "../thesis"
% TeX-engine: xetex
% End:

% !Mode:: "TeX:UTF-8"

\chapter{示例文档}[Example]

这是 \hithesis\ 的示例文档,基本上覆盖了模板中所有格式的设置。建议大家在使用模
板之前,除了阅读《\hithesis\:哈尔滨工业大学学位论文模板》\footnote{即
hithesis.pdf文件},本示例文档也最好能看一看。此示例文档尽量使用到所有的排版格式
,然而对于一些不在我工规范中规定的文档,理论上是由用户自由发挥,这里不给出样例
。需要另行载入的宏包和自定义命令在文件`hithesis.sty'中有示例,这里不列举。

\section{关于数字}[Number]

按《关于出版物上数字用法的试行规定》(1987年1月1日国家语言文字工作委员会等7个单位公布),除习惯用中文数字表示的以外,一般数字均用阿拉伯数字。
(1)公历的世纪、年代、年、月、日和时刻一律用阿拉伯数字,如20世纪,80年代,4时3刻等。年号要用四位数,如1989年,不能用89年。
(2)记数与计算(含正负整数、分数、小数、百分比、约数等)一律用阿拉伯数字,如3/4,4.5%,10个月,500多种等。
(3)一个数值的书写形式要照顾到上下文。不是出现在一组表示科学计量和具有统计意义数字中的一位数可以用汉字,如一个人,六条意见。星期几一律用汉字,如星期六。邻近两个数字并列连用,表示概数,应该用汉字数字,数字间不用顿号隔开,如三五天,七八十种,四十五六岁,一千七八百元等。
(4)数字作为词素构成定型的词、词组、惯用语、缩略语等应当使用汉字。如二倍体,三叶虫,第三世界,“七五”规划,相差十万八千里等。
(5)5位以上的数字,尾数零多的,可改写为以万、亿为单位的数。一般情况下不得以十、百、千、十万、百万、千万、十亿、百亿、千亿作为单位。如~\num{345000000}~公里可改写为3.45亿公里或~\num{34500}~万公里,但不能写为3亿~\num{4500}~万公里或3亿4千5百万公里。
(6)数字的书写不必每格一个数码,一般每两数码占一格,数字间分节不用分位号“,”,凡4位或4位以上的数都从个位起每3位数空半个数码(1/4汉字)。“\num{3000000}”,不要写成“3,000,000”,小数点后的数从小数点起向右按每三位一组分节。一个用阿拉伯数字书写的多位数不能从数字中间转行。
(7)数量的增加或减少要注意下列用词的概念:1)增加为(或增加到)过去的二倍,即过去为一,现在为二;2)增加(或增加了)二倍,即过去为一,现在为三;3)超额80%,即定额为100,现在为180;4)降低到80%,即过去为100,现在为80;5)降低(或降低了)80%,即原来为100,现在为20;6)为原数的1/4,即原数为4,现在为1,或原数为1,现在为0.25。
应特别注意在表达数字减小时,不宜用倍数,而应采用分数。如减少为原来的1/2,1/3等。


\section{索引示例}[Index]

为便于检索文中内容,可编制索引置于论文之后(根据需要决定是否设置)。索引以论文中
的专业词语为检索线索,指出其相关内容的所在页码。索引用中、英两种文字书写,中文在
前。\sindex[china]{qi!乔峰}\sindex[english]{Xu Zhu}\sindex[english]{Qiao Feng}
中文按各词汉语拼音第一个字母排序,英文按该词第一个英文字母排序。

\section{术语排版举例}[Glossaries and index]

术语的定义和使用可以结合索引,灵活使用。
例如,\gtssbp 是一种应用于狄利克雷过程抽样的算法。
下次出现将是另一种格式:\gtssbp 。
还可以切换单复数例如:\gscnas ,下次出现为:\gscnas 。
此处体现了\LaTeX\ 格式内容分离的优势。

\section{引用}[Cite]

\sindex[china]{du!段誉}引文标注遵照GB/T7714-2005,采用顺序编码制。正文中引用文献的标示应置于所引内容最后一个字的右上角,所引文献编号用阿拉伯数字置于方括号“[ ]”中,用小4号字体的上角标。要求:

(1)引用单篇文献时,如“二次铣削\cite{cnproceed}”。

(2)同一处引用多篇文献时,各篇文献的序号在方括号内全部列出,各序号间用“,”,如
遇连续序号,可标注讫序号。如,…形成了多种数学模型\cite{cnarticle,cnproceed}…
注意此处添加\cs{inlinecite}中文空格\inlinecite{cnarticle,cnproceed},可以在cfg文件中修改空格类型。

(3)多次引用同一文献时,在文献序号的“[ ]”后标注引文页码。如,…间质细胞CAMP含量
测定\cite[100-197]{cnarticle}…。…含量测定方法规定
\cite[92]{cnarticle}…。

(4)当提及的参考文献为文中直接说明时,则用小4号字与正文排齐,如“由文献\inlinecite{hithesis2017}可知”

\section{定理和定义等}[Theorem]
\begin{theorem}[\cite{cnproceed}]
宇宙大爆炸是一种爆炸。
\end{theorem}
\begin{definition}[(霍金)]
宇宙大爆炸是一种爆炸。
\end{definition}
\begin{assumption}
宇宙大爆炸是一种爆炸。
\end{assumption}
\begin{lemma}
宇宙大爆炸是一种爆炸。
\end{lemma}
\begin{corollary}
宇宙大爆炸是一种爆炸。
\end{corollary}
\begin{exercise}
宇宙大爆炸是一种爆炸。
\end{exercise}
\begin{problem}[(Albert Einstein)]
宇宙大爆炸是一种爆炸。
\end{problem}
\begin{remark}
宇宙大爆炸是一种爆炸。
\end{remark}
\begin{axiom}[(爱因斯坦)]
宇宙大爆炸是一种爆炸。
\end{axiom}
\begin{conjecture}
宇宙大爆炸是一种爆炸。
\end{conjecture}
\section{图片}[Pictures]
图应有自明性。插图应与文字紧密配合,文图相符,内容正确。选图要力求精练,插图、照
片应完整清晰。机械工程图:采用第一角投影法,严格按照GB4457~GB131-83《机械制图》
标准规定。数据流程图、程序流程图、系统流程图等按GB1526-89标准规定。电气图:图形
符号、文字符号等应符合附录3所列有关标准的规定。流程图:必须采用结构化程序并正确
运用流程框图。对无规定符号的图形应采用该行业的常用画法。坐标图的坐标线均用细实线
,粗细不得超过图中曲线;有数字标注的坐标图,必须注明坐标单位。照片图要求主题和主
要显示部分的轮廓鲜明,便于制版。如用放大或缩小的复制品,必须清晰,反差适中。照片
上应有表示目的物尺寸的标度。引用文献中的图时,除在正文文字中标注参考文献序号以外
,还必须在中、英文表题的右上角标注参考文献序号。

\subsection{博士毕业论文双语题注}[Doctoral picture example]
\begin{figure}[htpb]
\centering
\includegraphics[width = 0.4\textwidth]{golfer}
\bicaption[golfer1]{}{打高尔夫球球的人(博士论文双语题注)}{Fig.$\!$}{The person playing golf (Doctoral thesis)}
\end{figure}

每个图均应有图题(由图序和图名组成),图题不宜有标点符号,图名在图序之后空1个半
角字符排写。图序按章编排,如第1章第一个插图的图号为“图1-1”。图题置于图下,硕士论
文只用中文,博士论文用中、英两种文字,居中书写,中文在上,要求中文用宋体5号字,
英文用Times New Roman 5号字。有图注或其它说明时应置于图题之上。引用图应注明出处
,在图题右上角加引用文献号。图中若有分图时,分图题置于分图之下或图题之下,可以只
用中文书写,分图号用a)、b)等表示。图中各部分说明应采用中文(引用的外文图除外)或
数字符号,各项文字说明置于图题之上(有分图时,置于分图题之上)。图中文字用宋体、
Times New Roman字体,字号尽量采用5号字(当字数较多时可用小5号字,以清晰表达为原
则,但在一个插图内字号要统一)。同一图内使用文字应统一。图表中物理量、符号用斜体
。
\subsection{本硕论文题注}[Other picture example]
\begin{figure}[h]
\centering
\includegraphics[width = 0.4\textwidth]{golfer}
\caption{打高尔夫球的人,硕士论文要求只用汉语}
\end{figure}

\subsection{并排图和子图}[Abreast-picture and Sub-picture example]
\subsubsection{并排图}[Abreast-picture example]

使用并排图时,需要注意对齐方式。默认情况是中部对齐。这里给出中部对齐、顶部对齐
、图片底部对齐三种常见方式。其中,底部对齐方式有一个很巧妙的方式,将长度比较小
的图放在左面即可。

\begin{figure}[htbp]
\centering
\begin{minipage}{0.4\textwidth}
\centering
\includegraphics[width=\textwidth]{golfer}
\bicaption[golfer2]{}{打高尔夫球的人}{Fig.$\!$}{The person playing golf}
\end{minipage}
\centering
\begin{minipage}{0.4\textwidth}
\centering
\includegraphics[width=\textwidth]{golfer}
\bicaption[golfer3]{}{打高尔夫球的人。注意,这里默认居中}{Fig.$\!$}{The person playing golf. Please note that, it is vertically center aligned by default.}
\end{minipage}
\end{figure}

\begin{figure}[htbp]
\centering
\begin{minipage}[t]{0.4\textwidth}
\centering
\includegraphics[width=\textwidth]{golfer}
\bicaption[golfer5]{}{打高尔夫球的人}{Fig.$\!$}{The person playing golf}
\end{minipage}
\centering
\begin{minipage}[t]{0.4\textwidth}
\centering
\includegraphics[width=\textwidth]{golfer}
\bicaption[golfer8]{}{打高尔夫球的人。注意,此图是顶部对齐}{Fig.$\!$}{The person playing golf. Please note that, it is vertically top aligned.}
\end{minipage}
\end{figure}

\begin{figure}[htbp]
\centering
\begin{minipage}[t]{0.4\textwidth}
\centering
\includegraphics[width=\textwidth,height=\textwidth]{golfer}
\bicaption[golfer9]{}{打高尔夫球的人。注意,此图对齐方式是图片底部对齐}{Fig.$\!$}{The person playing golf. Please note that, it is vertically bottom aligned for figure.}
\end{minipage}
\centering
\begin{minipage}[t]{0.4\textwidth}
\centering
\includegraphics[width=\textwidth]{golfer}
\bicaption[golfer6]{}{打高尔夫球的人}{Fig.$\!$}{The person playing golf}
\end{minipage}
\end{figure}

\subsubsection{子图}[Sub-picture example]
注意:子图题注也可以只用中文。规范规定“分图题置于分图之下或图题之下”,但没有给出具体的格式要求。
没有要求的另外一个说法就是“无论什么格式都不对”。
所以只有在一个图中有标注“a),b)”,无法使用\cs{subfigure}的情况下,使用最后一个图例中的格式设置方法,否则不要使用。
为了应对“无论什么格式都不对”,这个子图图题使用“minipage”和“description”环境,宽度,对齐方式可以按照个人喜好自由设置,是否使用双语子图图题也可以自由设置。

\begin{figure}[!h]
\setlength{\subfigcapskip}{-1bp}
\centering
\begin{minipage}{\textwidth}
\centering
\subfigure{\label{golfer41}}\addtocounter{subfigure}{-2}
\subfigure[The person playing golf]{\subfigure[打高尔夫球的人~1]{\includegraphics[width=0.4\textwidth]{golfer}}}
\hspace{2em}
\subfigure{\label{golfer42}}\addtocounter{subfigure}{-2}
\subfigure[The person playing golf]{\subfigure[打高尔夫球的人~2]{\includegraphics[width=0.4\textwidth]{golfer}}}
\end{minipage}
\centering
\begin{minipage}{\textwidth}
\centering
\subfigure{\label{golfer43}}\addtocounter{subfigure}{-2}
\subfigure[The person playing golf]{\subfigure[打高尔夫球的人~3]{\includegraphics[width=0.4\textwidth]{golfer}}}
\hspace{2em}
\subfigure{\label{golfer44}}\addtocounter{subfigure}{-2}
\subfigure[The person playing golf. Here, 'hang indent' and 'center last line' are not stipulated in the regulation.]{\subfigure[打高尔夫球的人~4。注意,规范中没有明确规定要悬挂缩进、最后一行居中。]{\includegraphics[width=0.4\textwidth]{golfer}}}
\end{minipage}
\vspace{0.2em}
\bicaption[golfer4]{}{打高尔夫球的人}{Fig.$\!$}{The person playing gol}
\end{figure}

\begin{figure}[t]
  \centering
  \begin{minipage}{.7\linewidth}
    \setlength{\subfigcapskip}{-1bp}
    \centering
    \begin{minipage}{\textwidth}
      \centering
      \subfigure{\label{golfer45}}\addtocounter{subfigure}{-2}
      \subfigure[The person playing golf]{\subfigure[打高尔夫球的人~1]{\includegraphics[width=0.4\textwidth]{golfer}}}
      \hspace{4em}
      \subfigure{\label{golfer46}}\addtocounter{subfigure}{-2}
      \subfigure[The person playing golf]{\subfigure[打高尔夫球的人~2]{\includegraphics[width=0.4\textwidth]{golfer}}}
    \end{minipage}
    \vskip 0.2em
  \wuhao 注意:这里是中文图注添加位置(我工要求,图注在图题之上)。
    \vspace{0.2em}
\bicaption[golfer47]{}{打高尔夫球的人。注意,此处我工有另外一处要求,子图图题可以位于主图题之下。但由于没有明确说明位于下方具体是什么格式,所以这里不给出举例。}{Fig.$\!$}{The person playing golf. Please note that, although it is appropriate to put subfigures' captions under this caption as stipulated in regulation, but its format is not clearly stated.}
  \end{minipage}
\end{figure}

\begin{figure}[t]
\centering
\begin{tikzpicture}
	\node[anchor=south west,inner sep=0] (image) at (0,0) {\includegraphics[width=0.3\textwidth]{golfer}};
	\begin{scope}[x={(image.south east)},y={(image.north west)}]
		\node at (0.3,0.5) {a)};
		\node at (0.8,0.2) {b)};
	\end{scope}
\end{tikzpicture}
\bicaption[golfer0]{}{打高尔夫球球的人(博士论文双语题注)}{Fig.$\!$}{The person playing golf (Doctoral thesis)}
\vskip -0.4em
 \hspace{2em}
\begin{minipage}[t]{0.3\textwidth}
\wuhao \setlist[description]{font=\normalfont}
	\begin{description}
		\item[(a)]子图图题
	\end{description}
 \end{minipage}
 \hspace{2em}
 \begin{minipage}[t]{0.3\textwidth}
\wuhao \setlist[description]{font=\normalfont}
	\begin{description}
		\item[(b)]子图图题
		\item[(b)]Subfigure caption
	\end{description}
\end{minipage}
\end{figure}


\begin{figure}[!h]
	\centering
	\begin{sideways}
		\begin{minipage}{\textheight}
			\centering
			\fbox{\includegraphics[width=0.2\textwidth]{golfer}}
			\fbox{\includegraphics[width=0.2\textwidth]{golfer}}
			\fbox{\includegraphics[width=0.2\textwidth]{golfer}}
			\fbox{\includegraphics[width=0.2\textwidth]{golfer}}
			\fbox{\includegraphics[width=0.2\textwidth]{golfer}}
			\fbox{\includegraphics[width=0.2\textwidth]{golfer}}
			\fbox{\includegraphics[width=0.2\textwidth]{golfer}}
\bicaption[golfer7]{}{打高尔夫球的人(非规范要求)}{Fig.$\!$}{The person playing golf (Not stated in the regulation)}
		\end{minipage}
	\end{sideways}
\end{figure}

\clearpage

如果不想让图片浮动到下一章节,那么在此处使用\cs{clearpage}命令。

\section{如何做出符合规范的漂亮的图}
关于作图工具在后文\ref{drawtool}中给出一些作图工具的介绍,此处不多言。
此处以R语言和Tikz为例说明如何做出符合规范的图。

\subsection{Tikz作图举例}
使用Tikz作图核心思想是把格式、主题、样式与内容分离,定义在全局中。
注意字体设置可以有两种选择,如何字少,用五号字,字多用小五。
使用Tikz作图不会出现字体问题,字体会自动与正文一致。

\begin{figure}[thb!]
  \centering
      \begin{tikzpicture}[xscale=0.8,yscale=0.3,rotate=90]
        \small
	\draw (-22,6.5) node[refcell]{参考基因组};
	\draw[refline] (-23, 5) -- (27, 5);
	\draw (-22,3.75) node[tscell]{肿瘤样本};
	\draw (-20,3.75) node[tncell]{正常细胞};
	\draw[tnline] (-21, 2.5) -- (27, 2.5);
	\draw (-20,1.25) node[ttcell]{肿瘤细胞};
	\rcell{2}{6};
	\draw[fakeevolve] (4.5, 5.25) -- (4.5, 4.8);
	\ncell{2}{4};
	\draw[evolve] (4.5, 3) .. controls (4.5,2.8) and (-3.5,2.9) ..  (-3.5, 2);
	\draw[evolve] (4.5, 3) .. controls (4.5,2.8) and (11.5,2.9) .. (11.5, 2);
	\tcellone{-6}{1.5};
	\draw (-9, 2) node[ttcell]{1};
	\draw[evolve] (-3.5, 0) .. controls (-3.5,-0.2) and (-12,-0.1) .. (-12, -1.5);
	\draw[evolve] (-3.5, 0) .. controls (-3.5,-0.2) and (1.5,-0.1) .. (1.5, -1.5);
	\tcellthree{7}{1.5};
	\draw (4, 2) node[ttcell]{2};
	\draw[evolve] (11, 0.5) .. controls (11,0.3) and (19,0.4) .. (19, -1.5);
	\tcellfive{-16}{-2};
	\draw (-19, -1.5) node[ttcell]{3};
	\tcelltwo{-1}{-2};
	\draw (-4, -1.5) node[ttcell]{4};
	\tcellfour{12}{-2};
	\draw (9, -1.5) node[ttcell]{5};
      \end{tikzpicture}
  \begin{minipage}{.9\linewidth}
      \vskip 0.2em
      \wuhao 图中,带有箭头的淡蓝色箭头表示肿瘤子种群的进化方向。一般地,从肿瘤组织中取用于进行二代测序的样本中含有一定程度的正常细胞污染,因此肿瘤的样本中含有正常细胞和肿瘤细胞。每一个子种群的基因组的模拟过程是把生殖细胞变异和体细胞变异加入到参考基因组中。
      \vspace{0.6em}
  \end{minipage}
\bicaption[tumor]{}{肿瘤组织中各个子种群的进化示意图}{Fig.$\!$}{The diagram of tumor subpopulation evolution process}
\end{figure}

\subsection{R作图}
R是一种极具有代表性的典型的作图工具,应用广泛。
与Tikz图~\ref{tumor}~不同,R作图分两种情况:(1)可以转换为Tikz码;(2)不可转换为Tikz码。
第一种情况图形简单,图形中不含有很多数据点,使用R语言中的Tikz包即可。
第二种情况是图形复杂,含有海量数据点,这时候不要转成Tikz矢量图,这会使得论文体积巨大。
推荐使用pdf或png非矢量图形。
使用非矢量图形时要注意选择好字号(五号或小五),和字体(宋体、新罗马)然后选择生成图形大小,注意此时在正文中使用\cs{includegraphics}命令导入时,不要像导入矢量图那样控制图形大小,使用图形的原本的
宽度和高度,这样就确保了非矢量图形中的文字与正文一致了。

为了控制\hithesis\ 的大小,此处不给出具体举例,

\section{表格}

表应有自明性。表格不加左、右边线。表的编排建议采用国际通行的三线表。表中文字用宋
体~5~号字。每个表格均应有表题(由表序和表名组成)。表序一般按章编排,如第~1~章第
一个插表的序号为“表~1-1”等。表序与表名之间空一格,表名中不允许使用标点符号,表名
后不加标点。表题置于表上,硕士学位论文只用中文,博士学位论文用中、英文两种文字居
中排写,中文在上,要求中文用宋体~5~号字,英文用新罗马字体~5~号字。表头设计应简单
明了,尽量不用斜线。表头中可采用化学符号或物理量符号。


\subsection{普通表格的绘制方法}[Methods of drawing normal tables]

表格应具有三线表格式,因此需要调用~booktabs~宏包,其标准格式如表~\ref{table1}~所示。
\begin{table}[htbp]
\bicaption[table1]{}{符合研究生院绘图规范的表格}{Table$\!$}{Table in agreement of the standard from graduate school}
\vspace{0.5em}\centering\wuhao
\begin{tabular}{ccccc}
\toprule
$D$(in) & $P_u$(lbs) & $u_u$(in) & $\beta$ & $G_f$(psi.in)\\
\midrule
 5 & 269.8 & 0.000674 & 1.79 & 0.04089\\
10 & 421.0 & 0.001035 & 3.59 & 0.04089\\
20 & 640.2 & 0.001565 & 7.18 & 0.04089\\
\bottomrule
\end{tabular}
\end{table}
全表如用同一单位,则将单位符号移至表头右上角,加圆括号。表中数据应准确无误,书写
清楚。数字空缺的格内加横线“-”(占~2~个数字宽度)。表内文字或数字上、下或左、右
相同时,采用通栏处理方式,不允许用“〃”、“同上”之类的写法。表内文字说明,起行空一
格、转行顶格、句末不加标点。如某个表需要转页接排,在随后的各页上应重复表的编号。
编号后加“(续表)”,表题可省略。续表应重复表头。

\subsection{长表格的绘制方法}[Methods of drawing long tables]

长表格是当表格在当前页排不下而需要转页接排的情况下所采用的一种表格环境。若长表格
仍按照普通表格的绘制方法来获得,其所使用的\verb|table|浮动环境无法实现表格的换页
接排功能,表格下方过长部分会排在表格第1页的页脚以下。为了能够实现长表格的转页接
排功能,需要调用~longtable~宏包,由于长表格是跨页的文本内容,因此只需要单独的
\verb|longtable|环境,所绘制的长表格的格式如表~\ref{table2}~所示。

注意,长表格双语标题的格式。

\vspace{-1.5bp}
\ltfontsize{\wuhao[1.667]}
\wuhao[1.667]\begin{longtable}{ccc}%
\longbionenumcaption{}{{\wuhao 中国省级行政单位一览}\label{table2}}{Table$\!$}{}{{\wuhao Overview of the provincial administrative unit of China}}{-0.5em}{3.15bp}\\
%\caption{\wuhao 中国省级行政单位一览}\label{table2}\\
\toprule 名称 & 简称 & 省会或首府\\ \midrule
\endfirsthead
\multicolumn{3}{r}{表~\thetable(续表)}\vspace{0.5em}\\
\toprule 名称 & 简称 & 省会或首府\\ \midrule
\endhead
\midrule[0.5pt]
\endfoot
\bottomrule
\endlastfoot
北京市 & 京 & 北京\\
天津市 & 津 & 天津\\
河北省 & 冀 & 石家庄市\\
山西省 & 晋 & 太原市\\
内蒙古自治区 & 蒙 & 呼和浩特市\\
辽宁省 & 辽 & 沈阳市\\
吉林省 & 吉 & 长春市\\
黑龙江省 & 黑 & 哈尔滨市\\
上海市 & 沪/申 & 上海\\
江苏省 & 苏 & 南京市\\
浙江省 & 浙 & 杭州市\\
安徽省 & 皖 & 合肥市\\
福建省 & 闽 & 福州市\\
江西省 & 赣 & 南昌市\\
山东省 & 鲁 & 济南市\\
河南省 & 豫 & 郑州市\\
湖北省 & 鄂 & 武汉市\\
湖南省 & 湘 & 长沙市\\
广东省 & 粤 & 广州市\\
广西壮族自治区 & 桂 & 南宁市\\
海南省 & 琼 & 海口市\\
重庆市 & 渝 & 重庆\\
四川省 & 川/蜀 & 成都市\\
贵州省 & 黔/贵 & 贵阳市\\
云南省 & 云/滇 & 昆明市\\
西藏自治区 & 藏 & 拉萨市\\
陕西省 & 陕/秦 & 西安市\\
甘肃省 & 甘/陇 & 兰州市\\
青海省 & 青 & 西宁市\\
宁夏回族自治区 & 宁 & 银川市\\
新疆维吾尔自治区 & 新 & 乌鲁木齐市\\
香港特别行政区 & 港 & 香港\\
澳门特别行政区 & 澳 & 澳门\\
台湾省 & 台 & 台北市\\
\end{longtable}\normalsize
\vspace{-1em}

此长表格~\ref{table2}~第~2~页的标题“编号(续表)”和表头是通过代码自动添加上去的,无需人工添加,若表格在页面中的竖直位置发生了变化,长表格在第~2~页
及之后各页的标题和表头位置能够始终处于各页的最顶部,也无需人工调整,\LaTeX~系统的这一优点是~word~等软件所无法比拟的。

\subsection{列宽可调表格的绘制方法}[Methods of drawing tables with adjustable-width columns]
论文中能用到列宽可调表格的情况共有两种,一种是当插入的表格某一单元格内容过长以至
于一行放不下的情况,另一种是当对公式中首次出现的物理量符号进行注释的情况,这两种
情况都需要调用~tabularx~宏包。下面将分别对这两种情况下可调表格的绘制方法进行阐述
。
\subsubsection{表格内某单元格内容过长的情况}[The condition when the contents in
some cells of tables are too long]
首先给出这种情况下的一个例子如表~\ref{table3}~所示。
\begin{table}[htbp]
  \centering
\bicaption[table3]{}{最小的三个正整数的英文表示法}{Table$\!$}{The English construction of the smallest three positive integral numbers}\vspace{0.5em}\wuhao
\begin{tabularx}{0.7\textwidth}{llX}
\toprule
Value & Name & Alternate names, and names for sets of the given size\\
\midrule
1 & One & ace, single, singleton, unary, unit, unity\\
2 & Two & binary, brace, couple, couplet, distich, deuce, double, doubleton, duad, duality, duet, duo, dyad, pair, snake eyes, span, twain, twosome, yoke\\
3 & Three & deuce-ace, leash, set, tercet, ternary, ternion, terzetto, threesome, tierce, trey, triad, trine, trinity, trio, triplet, troika, hat-trick\\
\bottomrule
\end{tabularx}
\end{table}
tabularx环境共有两个必选参数:第1个参数用来确定表格的总宽度,第2个参数用来确定每
列格式,其中标为X的项表示该列的宽度可调,其宽度值由表格总宽度确定。标为X的列一般
选为单元格内容过长而无法置于一行的列,这样使得该列内容能够根据表格总宽度自动分行
。若列格式中存在不止一个X项,则这些标为X的列的列宽相同,因此,一般不将内容较短的
列设为X。标为X的列均为左对齐,因此其余列一般选为l(左对齐),这样可使得表格美观
,但也可以选为c或r。

\subsubsection{对物理量符号进行注释的情况}[The condition when physical symbols
need to be annotated]

为使得对公式中物理量符号注释的转行与破折号“———”后第一个字对齐,此处最好采用表格
环境。此表格无任何线条,左对齐,且在破折号处对齐,一共有“式中”二字、物理量符号和
注释三列,表格的总宽度可选为文本宽度,因此应该采用\verb|tabularx|环境。由
\verb|tabularx|环境生成的对公式中物理量符号进行注释的公式如式(\ref{eq:1})所示。
\begin{equation}\label{eq:1}
\ddot{\boldsymbol{\rho}}-\frac{\mu}{R_{t}^{3}}\left(3\mathbf{R_{t}}\frac{\mathbf{R_{t}\rho}}{R_{t}^{2}}-\boldsymbol{\rho}\right)=\mathbf{a}
\end{equation}
\begin{tabularx}{\textwidth}{@{}l@{\quad}r@{———}X@{}}
式中& $\boldsymbol{\rho}$ &追踪飞行器与目标飞行器之间的相对位置矢量;\\
&  $\boldsymbol{\ddot{\rho}}$&追踪飞行器与目标飞行器之间的相对加速度;\\
&  $\mathbf{a}$   &推力所产生的加速度;\\
&  $\mathbf{R_t}$ & 目标飞行器在惯性坐标系中的位置矢量;\\
&  $\omega_{t}$ & 目标飞行器的轨道角速度;\\
&  $\mathbf{g}$ & 重力加速度,$=\frac{\mu}{R_{t}^{3}}\left(
3\mathbf{R_{t}}\frac{\mathbf{R_{t}\rho}}{R_{t}^{2}}-\boldsymbol{\rho}\right)=\omega_{t}^{2}\frac{R_{t}}{p}\left(
3\mathbf{R_{t}}\frac{\mathbf{R_{t}\rho}}{R_{t}^{2}}-\boldsymbol{\rho}\right)$,这里~$p$~是目标飞行器的轨道半通径。
\end{tabularx}\vspace{3.15bp}
由此方法生成的注释内容应紧邻待注释公式并置于其下方,因此不能将代码放入
\verb|table|浮动环境中。但此方法不能实现自动转页接排,可能会在当前页剩余空间不够
时,全部移动到下一页而导致当前页出现很大空白。因此在需要转页处理时,还请您手动将
需要转页的代码放入一个新的\verb|tabularx|环境中,将原来的一个\verb|tabularx|环境
拆分为两个\verb|tabularx|环境。

\subsubsection{排版横版表格的举例}[An example of landscape table]

\begin{table}[p]
\centering
\begin{sideways}
\begin{minipage}{\textheight}
\bicaption[table4]{}{不在规范中规定的横版表格}{Table$\!$}{A table style which is not stated in the regulation}
\vspace{0.5em}\centering\wuhao
\begin{tabular}{ccccc}
\toprule
$D$(in) & $P_u$(lbs) & $u_u$(in) & $\beta$ & $G_f$(psi.in)\\
\midrule
 5 & 269.8 & 0.000674 & 1.79 & 0.04089\\
10 & 421.0 & 0.001035 & 3.59 & 0.04089\\
20 & 640.2 & 0.001565 & 7.18 & 0.04089\\
\bottomrule
\end{tabular}
\end{minipage}
\end{sideways}
\end{table}


\section{公式}
与正常\LaTeX\ 使用方法一致,此处略。关于公式中符号样式的定义在`hithesis.sty'有示
例。

\section{其他杂项}[Miscellaneous]

\subsection{右翻页}[Open right]

对于双面打印的论文,强制使每章的标题页出现右手边为右翻页。
规范中没有明确规定是否是右翻页打印。
模板给出了右翻页选项。
为了应对用户的个人喜好,在希望设置成右翻页的位置之前添加\cs{cleardoublepage}命令即可。

\subsection{算法}[Algorithms]
我工算法有以下几大特点。

(1)算法不在规范中要求。

(2)算法常常被使用(至少计算机学院)。

(3)格式乱,甚至出现了每个实验室的格式要求都不一样。

此处不给出示例,因为没法给,在
\href{https://github.com/dustincys/PlutoThesis}{https://github.com/dustincys/PlutoThesis}
的readme文件中有不同实验室算法要求说明。

\subsection{脚注}[Footnotes]
不在再规范\footnote{规范是指\PGR\ 和\UGR}中要求,模板默认使用清华大学的格式。

\subsection{源码}[Source code]
也不在再规范中要求。如果有需要最好使用minted包,但在编译的时候需要添加“
-shell-escape”选项且安装pygmentize软件,这些不在模板中默认载入,如果需要自行载入
。
\subsection{思源宋体}[Siyuan font]
如果要使用思源字体,需要思源字体的定义文件,此文件请到模板的开发版网址github:
\href{https://github.com/dustincys/hithesis}{https://github.com/dustincys/hithesis}
或者oschia:
\href{https://git.oschina.net/dustincys/hithesis}{https://git.oschina.net/dustincys/hithesis}
处下载。

\subsection{专业绘图工具}[Processional drawing tool]
\label{drawtool}
推荐使用tikz包,使用tikz源码绘图的好处是,图片中的字体与正文中的字体一致。具体如
何使用tikz绘图不属于模板范畴。
tikz适合用来画不需要大量实验数据支撑示意图。但R语言等专业绘图工具具有画出各种、
专业、复杂的数据图。R语言中有tikz包,能自动生成tikz码,这样tikz几乎无所不能。
对于排版有极致追求的小伙伴,可以参考
\href{http://www.texample.net/tikz/resources/}{http://www.texample.net/tikz/resources/}
中所列工具,几乎所有作图软件所作的图形都可转成tikz,然后可以自由的在tikz中修改
图中内容,定义字体等等。实现前文窝工规范中要求的图中字体的一致性的终极目标。


\subsection{术语词汇管理}[Manage glossaries]
推荐使用glossaries包管理术语、缩略语,可以自动生成首次全写,非首次缩写。

\subsection{\TeX\ 源码编辑器}[\TeX editor]
推荐:(1)付费软件Winedt;(2)免费软件kile;(3)vim或emaces或sublime等神级编
译器(需要配置)。

\subsection{\LaTeX\ 排版重要原则}[\LaTeX\ typesetting rules]
格式和内容分离是\LaTeX\ 最大优势,所有多次出现的内容、样式等等都可以定义为简单命
令、环境。这样的好处是方便修改、管理。例如,如果想要把所有的表示向量的符号由粗体
\cs{mathbf}变换到花体\cs{mathcal},只需修改该格式的命令的定义部分,不需要像MS
word那样处处修改。总而言之,使用自定义命令和环境才是正确的使用\LaTeX\ 的方式。

\section{关于捐助}
各位刀客和大侠如用的嗨,要解囊相助,请参照图~\ref{zfb}~中提示操作(二维码被矢量化后之后去
除了头像等冗余无用的部分~)。

\begin{figure}[!h]
\centering\includegraphics[width=0.4\textwidth]{zfb}
\vspace{0.2em}
\bicaption[Donation]{}{捐助,注意此处是子图只用汉语图题的形式,我工规定可以不用
英语图题}{Fig.$\!$}{Donation, please note that it is OK to use Chinese caption
only}
\end{figure}


% Local Variables:
% TeX-master: "../main"
% TeX-engine: xetex
% End:

\backmatter
% !Mode:: "TeX:UTF-8" 
\begin{conclusions}

学位论文的结论作为论文正文的最后一章单独排写,但不加章标题序号。

结论应是作者在学位论文研究过程中所取得的创新性成果的概要总结,不能与摘要混为一谈。博士学位论文结论应包括论文的主要结果、创新点、展望三部分,在结论中应概括论文的核心观点,明确、客观地指出本研究内容的创新性成果(含新见解、新观点、方法创新、技术创新、理论创新),并指出今后进一步在本研究方向进行研究工作的展望与设想。对所取得的创新性成果应注意从定性和定量两方面给出科学、准确的评价,分(1)、(2)、(3)…条列出,宜用“提出了”、“建立了”等词叙述。

\end{conclusions}
   % 结论
\bibliographystyle{gbt7714-numerical}
% \bibliographystyle{hitszthesis} % 深圳校区的同学请使用 hitszthesis 文献样式
%\bibliographystyle{hithesis} %理论上2020最新要求文献样式GB/T 7714—2015,但若院系要求文献英文作者不全大写,可改用hithesis文献样式
%%%%%%%%%%%%%%%%%%%%%%%%%%%%%%%%%%%%%%%%%%%%%%%%%%%%%%%%%%%%%%%%%%%%%%%%%%%%%%%%
%-- 注意:以下本硕博、博后书序不一致 --%
%%%%%%%%%%%%%%%%%%%%%%%%%%%%%%%%%%%%%%%%%%%%%%%%%%%%%%%%%%%%%%%%%%%%%%%%%%%%%%%%
% 本科书序(哈尔滨、深圳校区)
%%%%%%%%%%%%%%%%%%%%%%%%%%%%%%%%%%%%%%%%%%%%%%%%%%%%%%%%%%%%%%%%%%%%%%%%%%%%%%%%
\bibliography{reference} % 参考文献
\authorization %授权
% \authorization[scan.pdf] %添加扫描页的命令,与上互斥
% !Mode:: "TeX:UTF-8"
\begin{acknowledgements}
衷心感谢导师~XXX~教授对本人的精心指导。他的言传身教将使我终生受益。

……

感谢哈工大\LaTeX\ 论文模板\hithesis\ !

\end{acknowledgements}
 %致谢
\begin{appendix}%附录
\chapter{外文资料原文}
\label{cha:engorg}

\title{The title of the English paper}

\textbf{Abstract:} As one of the most widely used techniques in operations
research, \emph{ mathematical programming} is defined as a means of maximizing a
quantity known as \emph{bjective function}, subject to a set of constraints
represented by equations and inequalities. Some known subtopics of mathematical
programming are linear programming, nonlinear programming, multiobjective
programming, goal programming, dynamic programming, and multilevel
programming$^{[1]}$.

It is impossible to cover in a single chapter every concept of mathematical
programming. This chapter introduces only the basic concepts and techniques of
mathematical programming such that readers gain an understanding of them
throughout the book$^{[2,3]}$.


\section{Single-Objective Programming}
The general form of single-objective programming (SOP) is written
as follows,
\begin{equation}\tag*{(123)} % 如果附录中的公式不想让它出现在公式索引中,那就请
                             % 用 \tag*{xxxx}
\left\{\begin{array}{l}
\max \,\,f(x)\\[0.1 cm]
\mbox{subject to:} \\ [0.1 cm]
\qquad g_j(x)\le 0,\quad j=1,2,\cdots,p
\end{array}\right.
\end{equation}
which maximizes a real-valued function $f$ of
$x=(x_1,x_2,\cdots,x_n)$ subject to a set of constraints.

\newtheorem{mpdef}{Definition}[chapter]
\begin{mpdef}
In SOP, we call $x$ a decision vector, and
$x_1,x_2,\cdots,x_n$ decision variables. The function
$f$ is called the objective function. The set
\begin{equation}\tag*{(456)} % 这里同理,其它不再一一指定。
S=\left\{x\in\Re^n\bigm|g_j(x)\le 0,\,j=1,2,\cdots,p\right\}
\end{equation}
is called the feasible set. An element $x$ in $S$ is called a
feasible solution.
\end{mpdef}

\newtheorem{mpdefop}[mpdef]{Definition}
\begin{mpdefop}
A feasible solution $x^*$ is called the optimal
solution of SOP if and only if
\begin{equation}
f(x^*)\ge f(x)
\end{equation}
for any feasible solution $x$.
\end{mpdefop}

One of the outstanding contributions to mathematical programming was known as
the Kuhn-Tucker conditions\ref{eq:ktc}. In order to introduce them, let us give
some definitions. An inequality constraint $g_j(x)\le 0$ is said to be active at
a point $x^*$ if $g_j(x^*)=0$. A point $x^*$ satisfying $g_j(x^*)\le 0$ is said
to be regular if the gradient vectors $\nabla g_j(x)$ of all active constraints
are linearly independent.

Let $x^*$ be a regular point of the constraints of SOP and assume that all the
functions $f(x)$ and $g_j(x),j=1,2,\cdots,p$ are differentiable. If $x^*$ is a
local optimal solution, then there exist Lagrange multipliers
$\lambda_j,j=1,2,\cdots,p$ such that the following Kuhn-Tucker conditions hold,
\begin{equation}
\label{eq:ktc}
\left\{\begin{array}{l}
    \nabla f(x^*)-\sum\limits_{j=1}^p\lambda_j\nabla g_j(x^*)=0\\[0.3cm]
    \lambda_jg_j(x^*)=0,\quad j=1,2,\cdots,p\\[0.2cm]
    \lambda_j\ge 0,\quad j=1,2,\cdots,p.
\end{array}\right.
\end{equation}
If all the functions $f(x)$ and $g_j(x),j=1,2,\cdots,p$ are convex and
differentiable, and the point $x^*$ satisfies the Kuhn-Tucker conditions
(\ref{eq:ktc}), then it has been proved that the point $x^*$ is a global optimal
solution of SOP.

\subsection{Linear Programming}
\label{sec:lp}

If the functions $f(x),g_j(x),j=1,2,\cdots,p$ are all linear, then SOP is called
a {\em linear programming}.

The feasible set of linear is always convex. A point $x$ is called an extreme
point of convex set $S$ if $x\in S$ and $x$ cannot be expressed as a convex
combination of two points in $S$. It has been shown that the optimal solution to
linear programming corresponds to an extreme point of its feasible set provided
that the feasible set $S$ is bounded. This fact is the basis of the {\em simplex
  algorithm} which was developed by Dantzig as a very efficient method for
solving linear programming.
\begin{table}[ht]
\centering
  \centering
  \caption*{Table~1\hskip1em This is an example for manually numbered table, which
    would not appear in the list of tables}
  \label{tab:badtabular2}
  \begin{tabular}[c]{|m{1.5cm}|c|c|c|c|c|c|}\hline
    \multicolumn{2}{|c|}{Network Topology} & \# of nodes &
    \multicolumn{3}{c|}{\# of clients} & Server \\\hline
    GT-ITM & Waxman Transit-Stub & 600 &
    \multirow{2}{2em}{2\%}&
    \multirow{2}{2em}{10\%}&
    \multirow{2}{2em}{50\%}&
    \multirow{2}{1.2in}{Max. Connectivity}\\\cline{1-3}
    \multicolumn{2}{|c|}{Inet-2.1} & 6000 & & & &\\\hline
    & \multicolumn{2}{c|}{ABCDEF} &\multicolumn{4}{c|}{} \\\hline
\end{tabular}
\end{table}

Roughly speaking, the simplex algorithm examines only the extreme points of the
feasible set, rather than all feasible points. At first, the simplex algorithm
selects an extreme point as the initial point. The successive extreme point is
selected so as to improve the objective function value. The procedure is
repeated until no improvement in objective function value can be made. The last
extreme point is the optimal solution.

\subsection{Nonlinear Programming}

If at least one of the functions $f(x),g_j(x),j=1,2,\cdots,p$ is nonlinear, then
SOP is called a {\em nonlinear programming}.

A large number of classical optimization methods have been developed to treat
special-structural nonlinear programming based on the mathematical theory
concerned with analyzing the structure of problems.

Now we consider a nonlinear programming which is confronted solely with
maximizing a real-valued function with domain $\Re^n$.  Whether derivatives are
available or not, the usual strategy is first to select a point in $\Re^n$ which
is thought to be the most likely place where the maximum exists. If there is no
information available on which to base such a selection, a point is chosen at
random. From this first point an attempt is made to construct a sequence of
points, each of which yields an improved objective function value over its
predecessor. The next point to be added to the sequence is chosen by analyzing
the behavior of the function at the previous points. This construction continues
until some termination criterion is met. Methods based upon this strategy are
called {\em ascent methods}, which can be classified as {\em direct methods},
{\em gradient methods}, and {\em Hessian methods} according to the information
about the behavior of objective function $f$. Direct methods require only that
the function can be evaluated at each point. Gradient methods require the
evaluation of first derivatives of $f$. Hessian methods require the evaluation
of second derivatives. In fact, there is no superior method for all
problems. The efficiency of a method is very much dependent upon the objective
function.

\subsection{Integer Programming}

{\em Integer programming} is a special mathematical programming in which all of
the variables are assumed to be only integer values. When there are not only
integer variables but also conventional continuous variables, we call it {\em
  mixed integer programming}. If all the variables are assumed either 0 or 1,
then the problem is termed a {\em zero-one programming}. Although integer
programming can be solved by an {\em exhaustive enumeration} theoretically, it
is impractical to solve realistically sized integer programming problems. The
most successful algorithm so far found to solve integer programming is called
the {\em branch-and-bound enumeration} developed by Balas (1965) and Dakin
(1965). The other technique to integer programming is the {\em cutting plane
  method} developed by Gomory (1959).

\hfill\textit{Uncertain Programming\/}\quad(\textsl{BaoDing Liu, 2006.2})

\section*{References}
\noindent{\itshape NOTE: These references are only for demonstration. They are
  not real citations in the original text.}

\begin{translationbib}
\item Donald E. Knuth. The \TeX book. Addison-Wesley, 1984. ISBN: 0-201-13448-9
\item Paul W. Abrahams, Karl Berry and Kathryn A. Hargreaves. \TeX\ for the
  Impatient. Addison-Wesley, 1990. ISBN: 0-201-51375-7
\item David Salomon. The advanced \TeX book.  New York : Springer, 1995. ISBN:0-387-94556-3
\end{translationbib}

\chapter{外文资料的调研阅读报告或书面翻译}

\title{英文资料的中文标题}

{\heiti 摘要:} 本章为外文资料翻译内容。如果有摘要可以直接写上来,这部分好像没有
明确的规定。

\section{单目标规划}
北冥有鱼,其名为鲲。鲲之大,不知其几千里也。化而为鸟,其名为鹏。鹏之背,不知其几
千里也。怒而飞,其翼若垂天之云。是鸟也,海运则将徙于南冥。南冥者,天池也。
\begin{equation}\tag*{(123)}
 p(y|\mathbf{x}) = \frac{p(\mathbf{x},y)}{p(\mathbf{x})}=
\frac{p(\mathbf{x}|y)p(y)}{p(\mathbf{x})}
\end{equation}

吾生也有涯,而知也无涯。以有涯随无涯,殆已!已而为知者,殆而已矣!为善无近名,为
恶无近刑,缘督以为经,可以保身,可以全生,可以养亲,可以尽年。

\subsection{线性规划}
庖丁为文惠君解牛,手之所触,肩之所倚,足之所履,膝之所倚,砉然响然,奏刀騞然,莫
不中音,合于桑林之舞,乃中经首之会。
\begin{table}[ht]
\centering
  \centering
  \caption*{表~1\hskip1em 这是手动编号但不出现在索引中的一个表格例子}
  \label{tab:badtabular3}
  \begin{tabular}[c]{|m{1.5cm}|c|c|c|c|c|c|}\hline
    \multicolumn{2}{|c|}{Network Topology} & \# of nodes &
    \multicolumn{3}{c|}{\# of clients} & Server \\\hline
    GT-ITM & Waxman Transit-Stub & 600 &
    \multirow{2}{2em}{2\%}&
    \multirow{2}{2em}{10\%}&
    \multirow{2}{2em}{50\%}&
    \multirow{2}{1.2in}{Max. Connectivity}\\\cline{1-3}
    \multicolumn{2}{|c|}{Inet-2.1} & 6000 & & & &\\\hline
    & \multicolumn{2}{c|}{ABCDEF} &\multicolumn{4}{c|}{} \\\hline
\end{tabular}
\end{table}

文惠君曰:“嘻,善哉!技盖至此乎?”庖丁释刀对曰:“臣之所好者道也,进乎技矣。始臣之
解牛之时,所见无非全牛者;三年之后,未尝见全牛也;方今之时,臣以神遇而不以目视,
官知止而神欲行。依乎天理,批大郤,导大窾,因其固然。技经肯綮之未尝,而况大坬乎!
良庖岁更刀,割也;族庖月更刀,折也;今臣之刀十九年矣,所解数千牛矣,而刀刃若新发
于硎。彼节者有间而刀刃者无厚,以无厚入有间,恢恢乎其于游刃必有余地矣。是以十九年
而刀刃若新发于硎。虽然,每至于族,吾见其难为,怵然为戒,视为止,行为迟,动刀甚微,
謋然已解,如土委地。提刀而立,为之而四顾,为之踌躇满志,善刀而藏之。”

文惠君曰:“善哉!吾闻庖丁之言,得养生焉。”


\subsection{非线性规划}
孔子与柳下季为友,柳下季之弟名曰盗跖。盗跖从卒九千人,横行天下,侵暴诸侯。穴室枢
户,驱人牛马,取人妇女。贪得忘亲,不顾父母兄弟,不祭先祖。所过之邑,大国守城,小
国入保,万民苦之。孔子谓柳下季曰:“夫为人父者,必能诏其子;为人兄者,必能教其弟。
若父不能诏其子,兄不能教其弟,则无贵父子兄弟之亲矣。今先生,世之才士也,弟为盗
跖,为天下害,而弗能教也,丘窃为先生羞之。丘请为先生往说之。”

柳下季曰:“先生言为人父者必能诏其子,为人兄者必能教其弟,若子不听父之诏,弟不受
兄之教,虽今先生之辩,将奈之何哉?且跖之为人也,心如涌泉,意如飘风,强足以距敌,
辩足以饰非。顺其心则喜,逆其心则怒,易辱人以言。先生必无往。”

孔子不听,颜回为驭,子贡为右,往见盗跖。

\subsection{整数规划}
盗跖乃方休卒徒大山之阳,脍人肝而餔之。孔子下车而前,见谒者曰:“鲁人孔丘,闻将军
高义,敬再拜谒者。”谒者入通。盗跖闻之大怒,目如明星,发上指冠,曰:“此夫鲁国之
巧伪人孔丘非邪?为我告之:尔作言造语,妄称文、武,冠枝木之冠,带死牛之胁,多辞缪
说,不耕而食,不织而衣,摇唇鼓舌,擅生是非,以迷天下之主,使天下学士不反其本,妄
作孝弟,而侥幸于封侯富贵者也。子之罪大极重,疾走归!不然,我将以子肝益昼餔之膳。”


\chapter{其它附录}
前面两个附录主要是给本科生做例子。其它附录的内容可以放到这里,当然如果你愿意,可
以把这部分也放到独立的文件中,然后将其到主文件中。
%本科生翻译论文
\end{appendix}
%%%%%%%%%%%%%%%%%%%%%%%%%%%%%%%%%%%%%%%%%%%%%%%%%%%%%%%%%%%%%%%%%%%%%%%%%%%%%%%%
% 本科书序(威海校区)
%%%%%%%%%%%%%%%%%%%%%%%%%%%%%%%%%%%%%%%%%%%%%%%%%%%%%%%%%%%%%%%%%%%%%%%%%%%%%%%%
% \authorization %授权
% % \authorization[scan.pdf] %添加扫描页的命令,与上互斥
% \bibliography{reference} % 参考文献
% % !Mode:: "TeX:UTF-8"
\begin{acknowledgements}
衷心感谢导师~XXX~教授对本人的精心指导。他的言传身教将使我终生受益。

……

感谢哈工大\LaTeX\ 论文模板\hithesis\ !

\end{acknowledgements}
 %致谢
% \begin{appendix}%附录
% \chapter{外文资料原文}
\label{cha:engorg}

\title{The title of the English paper}

\textbf{Abstract:} As one of the most widely used techniques in operations
research, \emph{ mathematical programming} is defined as a means of maximizing a
quantity known as \emph{bjective function}, subject to a set of constraints
represented by equations and inequalities. Some known subtopics of mathematical
programming are linear programming, nonlinear programming, multiobjective
programming, goal programming, dynamic programming, and multilevel
programming$^{[1]}$.

It is impossible to cover in a single chapter every concept of mathematical
programming. This chapter introduces only the basic concepts and techniques of
mathematical programming such that readers gain an understanding of them
throughout the book$^{[2,3]}$.


\section{Single-Objective Programming}
The general form of single-objective programming (SOP) is written
as follows,
\begin{equation}\tag*{(123)} % 如果附录中的公式不想让它出现在公式索引中,那就请
                             % 用 \tag*{xxxx}
\left\{\begin{array}{l}
\max \,\,f(x)\\[0.1 cm]
\mbox{subject to:} \\ [0.1 cm]
\qquad g_j(x)\le 0,\quad j=1,2,\cdots,p
\end{array}\right.
\end{equation}
which maximizes a real-valued function $f$ of
$x=(x_1,x_2,\cdots,x_n)$ subject to a set of constraints.

\newtheorem{mpdef}{Definition}[chapter]
\begin{mpdef}
In SOP, we call $x$ a decision vector, and
$x_1,x_2,\cdots,x_n$ decision variables. The function
$f$ is called the objective function. The set
\begin{equation}\tag*{(456)} % 这里同理,其它不再一一指定。
S=\left\{x\in\Re^n\bigm|g_j(x)\le 0,\,j=1,2,\cdots,p\right\}
\end{equation}
is called the feasible set. An element $x$ in $S$ is called a
feasible solution.
\end{mpdef}

\newtheorem{mpdefop}[mpdef]{Definition}
\begin{mpdefop}
A feasible solution $x^*$ is called the optimal
solution of SOP if and only if
\begin{equation}
f(x^*)\ge f(x)
\end{equation}
for any feasible solution $x$.
\end{mpdefop}

One of the outstanding contributions to mathematical programming was known as
the Kuhn-Tucker conditions\ref{eq:ktc}. In order to introduce them, let us give
some definitions. An inequality constraint $g_j(x)\le 0$ is said to be active at
a point $x^*$ if $g_j(x^*)=0$. A point $x^*$ satisfying $g_j(x^*)\le 0$ is said
to be regular if the gradient vectors $\nabla g_j(x)$ of all active constraints
are linearly independent.

Let $x^*$ be a regular point of the constraints of SOP and assume that all the
functions $f(x)$ and $g_j(x),j=1,2,\cdots,p$ are differentiable. If $x^*$ is a
local optimal solution, then there exist Lagrange multipliers
$\lambda_j,j=1,2,\cdots,p$ such that the following Kuhn-Tucker conditions hold,
\begin{equation}
\label{eq:ktc}
\left\{\begin{array}{l}
    \nabla f(x^*)-\sum\limits_{j=1}^p\lambda_j\nabla g_j(x^*)=0\\[0.3cm]
    \lambda_jg_j(x^*)=0,\quad j=1,2,\cdots,p\\[0.2cm]
    \lambda_j\ge 0,\quad j=1,2,\cdots,p.
\end{array}\right.
\end{equation}
If all the functions $f(x)$ and $g_j(x),j=1,2,\cdots,p$ are convex and
differentiable, and the point $x^*$ satisfies the Kuhn-Tucker conditions
(\ref{eq:ktc}), then it has been proved that the point $x^*$ is a global optimal
solution of SOP.

\subsection{Linear Programming}
\label{sec:lp}

If the functions $f(x),g_j(x),j=1,2,\cdots,p$ are all linear, then SOP is called
a {\em linear programming}.

The feasible set of linear is always convex. A point $x$ is called an extreme
point of convex set $S$ if $x\in S$ and $x$ cannot be expressed as a convex
combination of two points in $S$. It has been shown that the optimal solution to
linear programming corresponds to an extreme point of its feasible set provided
that the feasible set $S$ is bounded. This fact is the basis of the {\em simplex
  algorithm} which was developed by Dantzig as a very efficient method for
solving linear programming.
\begin{table}[ht]
\centering
  \centering
  \caption*{Table~1\hskip1em This is an example for manually numbered table, which
    would not appear in the list of tables}
  \label{tab:badtabular2}
  \begin{tabular}[c]{|m{1.5cm}|c|c|c|c|c|c|}\hline
    \multicolumn{2}{|c|}{Network Topology} & \# of nodes &
    \multicolumn{3}{c|}{\# of clients} & Server \\\hline
    GT-ITM & Waxman Transit-Stub & 600 &
    \multirow{2}{2em}{2\%}&
    \multirow{2}{2em}{10\%}&
    \multirow{2}{2em}{50\%}&
    \multirow{2}{1.2in}{Max. Connectivity}\\\cline{1-3}
    \multicolumn{2}{|c|}{Inet-2.1} & 6000 & & & &\\\hline
    & \multicolumn{2}{c|}{ABCDEF} &\multicolumn{4}{c|}{} \\\hline
\end{tabular}
\end{table}

Roughly speaking, the simplex algorithm examines only the extreme points of the
feasible set, rather than all feasible points. At first, the simplex algorithm
selects an extreme point as the initial point. The successive extreme point is
selected so as to improve the objective function value. The procedure is
repeated until no improvement in objective function value can be made. The last
extreme point is the optimal solution.

\subsection{Nonlinear Programming}

If at least one of the functions $f(x),g_j(x),j=1,2,\cdots,p$ is nonlinear, then
SOP is called a {\em nonlinear programming}.

A large number of classical optimization methods have been developed to treat
special-structural nonlinear programming based on the mathematical theory
concerned with analyzing the structure of problems.

Now we consider a nonlinear programming which is confronted solely with
maximizing a real-valued function with domain $\Re^n$.  Whether derivatives are
available or not, the usual strategy is first to select a point in $\Re^n$ which
is thought to be the most likely place where the maximum exists. If there is no
information available on which to base such a selection, a point is chosen at
random. From this first point an attempt is made to construct a sequence of
points, each of which yields an improved objective function value over its
predecessor. The next point to be added to the sequence is chosen by analyzing
the behavior of the function at the previous points. This construction continues
until some termination criterion is met. Methods based upon this strategy are
called {\em ascent methods}, which can be classified as {\em direct methods},
{\em gradient methods}, and {\em Hessian methods} according to the information
about the behavior of objective function $f$. Direct methods require only that
the function can be evaluated at each point. Gradient methods require the
evaluation of first derivatives of $f$. Hessian methods require the evaluation
of second derivatives. In fact, there is no superior method for all
problems. The efficiency of a method is very much dependent upon the objective
function.

\subsection{Integer Programming}

{\em Integer programming} is a special mathematical programming in which all of
the variables are assumed to be only integer values. When there are not only
integer variables but also conventional continuous variables, we call it {\em
  mixed integer programming}. If all the variables are assumed either 0 or 1,
then the problem is termed a {\em zero-one programming}. Although integer
programming can be solved by an {\em exhaustive enumeration} theoretically, it
is impractical to solve realistically sized integer programming problems. The
most successful algorithm so far found to solve integer programming is called
the {\em branch-and-bound enumeration} developed by Balas (1965) and Dakin
(1965). The other technique to integer programming is the {\em cutting plane
  method} developed by Gomory (1959).

\hfill\textit{Uncertain Programming\/}\quad(\textsl{BaoDing Liu, 2006.2})

\section*{References}
\noindent{\itshape NOTE: These references are only for demonstration. They are
  not real citations in the original text.}

\begin{translationbib}
\item Donald E. Knuth. The \TeX book. Addison-Wesley, 1984. ISBN: 0-201-13448-9
\item Paul W. Abrahams, Karl Berry and Kathryn A. Hargreaves. \TeX\ for the
  Impatient. Addison-Wesley, 1990. ISBN: 0-201-51375-7
\item David Salomon. The advanced \TeX book.  New York : Springer, 1995. ISBN:0-387-94556-3
\end{translationbib}

\chapter{外文资料的调研阅读报告或书面翻译}

\title{英文资料的中文标题}

{\heiti 摘要:} 本章为外文资料翻译内容。如果有摘要可以直接写上来,这部分好像没有
明确的规定。

\section{单目标规划}
北冥有鱼,其名为鲲。鲲之大,不知其几千里也。化而为鸟,其名为鹏。鹏之背,不知其几
千里也。怒而飞,其翼若垂天之云。是鸟也,海运则将徙于南冥。南冥者,天池也。
\begin{equation}\tag*{(123)}
 p(y|\mathbf{x}) = \frac{p(\mathbf{x},y)}{p(\mathbf{x})}=
\frac{p(\mathbf{x}|y)p(y)}{p(\mathbf{x})}
\end{equation}

吾生也有涯,而知也无涯。以有涯随无涯,殆已!已而为知者,殆而已矣!为善无近名,为
恶无近刑,缘督以为经,可以保身,可以全生,可以养亲,可以尽年。

\subsection{线性规划}
庖丁为文惠君解牛,手之所触,肩之所倚,足之所履,膝之所倚,砉然响然,奏刀騞然,莫
不中音,合于桑林之舞,乃中经首之会。
\begin{table}[ht]
\centering
  \centering
  \caption*{表~1\hskip1em 这是手动编号但不出现在索引中的一个表格例子}
  \label{tab:badtabular3}
  \begin{tabular}[c]{|m{1.5cm}|c|c|c|c|c|c|}\hline
    \multicolumn{2}{|c|}{Network Topology} & \# of nodes &
    \multicolumn{3}{c|}{\# of clients} & Server \\\hline
    GT-ITM & Waxman Transit-Stub & 600 &
    \multirow{2}{2em}{2\%}&
    \multirow{2}{2em}{10\%}&
    \multirow{2}{2em}{50\%}&
    \multirow{2}{1.2in}{Max. Connectivity}\\\cline{1-3}
    \multicolumn{2}{|c|}{Inet-2.1} & 6000 & & & &\\\hline
    & \multicolumn{2}{c|}{ABCDEF} &\multicolumn{4}{c|}{} \\\hline
\end{tabular}
\end{table}

文惠君曰:“嘻,善哉!技盖至此乎?”庖丁释刀对曰:“臣之所好者道也,进乎技矣。始臣之
解牛之时,所见无非全牛者;三年之后,未尝见全牛也;方今之时,臣以神遇而不以目视,
官知止而神欲行。依乎天理,批大郤,导大窾,因其固然。技经肯綮之未尝,而况大坬乎!
良庖岁更刀,割也;族庖月更刀,折也;今臣之刀十九年矣,所解数千牛矣,而刀刃若新发
于硎。彼节者有间而刀刃者无厚,以无厚入有间,恢恢乎其于游刃必有余地矣。是以十九年
而刀刃若新发于硎。虽然,每至于族,吾见其难为,怵然为戒,视为止,行为迟,动刀甚微,
謋然已解,如土委地。提刀而立,为之而四顾,为之踌躇满志,善刀而藏之。”

文惠君曰:“善哉!吾闻庖丁之言,得养生焉。”


\subsection{非线性规划}
孔子与柳下季为友,柳下季之弟名曰盗跖。盗跖从卒九千人,横行天下,侵暴诸侯。穴室枢
户,驱人牛马,取人妇女。贪得忘亲,不顾父母兄弟,不祭先祖。所过之邑,大国守城,小
国入保,万民苦之。孔子谓柳下季曰:“夫为人父者,必能诏其子;为人兄者,必能教其弟。
若父不能诏其子,兄不能教其弟,则无贵父子兄弟之亲矣。今先生,世之才士也,弟为盗
跖,为天下害,而弗能教也,丘窃为先生羞之。丘请为先生往说之。”

柳下季曰:“先生言为人父者必能诏其子,为人兄者必能教其弟,若子不听父之诏,弟不受
兄之教,虽今先生之辩,将奈之何哉?且跖之为人也,心如涌泉,意如飘风,强足以距敌,
辩足以饰非。顺其心则喜,逆其心则怒,易辱人以言。先生必无往。”

孔子不听,颜回为驭,子贡为右,往见盗跖。

\subsection{整数规划}
盗跖乃方休卒徒大山之阳,脍人肝而餔之。孔子下车而前,见谒者曰:“鲁人孔丘,闻将军
高义,敬再拜谒者。”谒者入通。盗跖闻之大怒,目如明星,发上指冠,曰:“此夫鲁国之
巧伪人孔丘非邪?为我告之:尔作言造语,妄称文、武,冠枝木之冠,带死牛之胁,多辞缪
说,不耕而食,不织而衣,摇唇鼓舌,擅生是非,以迷天下之主,使天下学士不反其本,妄
作孝弟,而侥幸于封侯富贵者也。子之罪大极重,疾走归!不然,我将以子肝益昼餔之膳。”


\chapter{其它附录}
前面两个附录主要是给本科生做例子。其它附录的内容可以放到这里,当然如果你愿意,可
以把这部分也放到独立的文件中,然后将其到主文件中。
%本科生翻译论文
% \end{appendix}
%%%%%%%%%%%%%%%%%%%%%%%%%%%%%%%%%%%%%%%%%%%%%%%%%%%%%%%%%%%%%%%%%%%%%%%%%%%%%%%%
% 硕博书序
%%%%%%%%%%%%%%%%%%%%%%%%%%%%%%%%%%%%%%%%%%%%%%%%%%%%%%%%%%%%%%%%%%%%%%%%%%%%%%%%
% \bibliography{reference} % 参考文献
% \begin{appendix}%附录
% % -*-coding: utf-8 -*-
%%%%%%%%%%%%%%%%%%%%%%%%%%%%%%%%%%%%%%%%%%%%%%%%%%%%%%%%%
\chapter{带章节的附录}[Full Appendix]%
完整的附录内容,包含章节,公式,图表等

%%%%%%%%%%%%%%%%%%%%%%%%%%%%%%%%%%%%%%%%%%%%%%%%%%%%%%%%%
\section{附录节的内容}[Section in Appendix]
这是附录的节的内容

附录中图的示例:
\begin{figure}[htbp]
\centering
\includegraphics[width = 0.4\textwidth]{golfer}
%\bicaption[golfer5]{}{\xiaosi[0]打高尔夫球的人}{Fig.$\!$}{The person playing golf}\vspace{-1em}
\caption{\xiaosi[0]打高尔夫球的人}
\end{figure}

附录中公式的示例:
\begin{align}
a & = b \times c \\
E & = m c^2
\label{eq}
\end{align}

\chapter{这个星球上最好的免费Linux软件列表}[List of the Best Linux Software in our Planet]
\section{系统}

\href{http://fvwm.org/}{FVWM 自从上世纪诞生以来,此星球最强大的窗口管理器。}
推荐基于FVWM的桌面设计hifvwm:\href{https://github.com/dustincys/hifvwm}{https://github.com/dustincys/hifvwm}。

\subsection{hifvwm的优点}

\begin{enumerate}
	\item 即使打开上百个窗口也不会“蒙圈”。计算机性能越来越强大,窗口任务的管理必须要升级到打怪兽级别。
	\item 自动同步Bing搜索主页的壁纸。每次电脑开机,午夜零点自动更新,用户
		也可以手动更新,从此审美再也不疲劳。
	\item 切换窗口自动聚焦到最上面的窗口。使用键盘快捷键切换窗口时候,减少
		操作过程,自动聚焦到目标窗口。这一特性是虚拟窗口必须的人性化设
		计。
	\item 类似window右下角的功能的最小化窗口来显示桌面的功能此处类似
		win7/win10,实现在一个桌面之内操作多个任务。
	\item 任务栏结合标题栏。采用任务栏和标题栏结合,节省空间。
	\item 同类窗口切换。可以在同类窗口之内类似alt-tab的方式切换。
	\item ……
\end{enumerate}

\section{其他}

\href{https://github.com/goldendict/goldendict}{goldendict 星球最强大的桌面字典。}

\href{https://github.com/yarrick/iodine}{iodine,“HIT-WLAN + 锐捷”时代的福音。}

\href{http://www.aircrack-ng.org/}{aircrack,Wifi“安全性评估”工具。}

\href{https://www.ledger-cli.org/}{ledger,前“金融区块链”时代最好的复式记账系统。}

\href{https://orgmode.org/}{orgmode,最强大的笔记系统,从来没有之一。}

\href{https://www.jianguoyun.com/}{坚果云,国内一款支持WebDav的云盘系统,国内真正的云盘没有之一。}

\href{http://www.mutt.org/}{mutt, ``All mail clients suck. This one just sucks less.''}

\section{vim}
实现中英文每一句一行,以及实现每一句折叠断行的简单正则式,tex源码更加乖乖。
\begin{lstlisting}
vnoremap <leader>fae J:s/[.!?]\zs\s\+/\="\r".matchstr(getline('.'), '^\s*')/g<CR>
vnoremap <leader>fac J:s/[。!?]/\=submatch(0)."\n".matchstr(getline('.'), '^\s*')/g<CR>
vnoremap <leader>fle :!fmt -80 -s<CR>
\end{lstlisting}

% \end{appendix}
% % !Mode:: "TeX:UTF-8" 
\begin{publication}
\noindent\textbf{发表的相关论文}
\begin{publist}
\item	XXX,XXX. Static Oxidation Model of Al-Mg/C Dissipation Thermal Protection Materials[J]. Rare Metal Materials and Engineering, 2010, 39(Suppl. 1): 520-524.(SCI~收录,IDS号为~669JS,IF=0.16)
\item XXX,XXX. 精密超声振动切削单晶铜的计算机仿真研究[J]. 系统仿真学报,2007,19(4):738-741,753.(EI~收录号:20071310514841)
\item XXX,XXX. 局部多孔质气体静压轴向轴承静态特性的数值求解[J]. 摩擦学学报,2007(1):68-72.(EI~收录号:20071510544816)
\item XXX,XXX. 硬脆光学晶体材料超精密切削理论研究综述[J]. 机械工程学报,2003,39(8):15-22.(EI~收录号:2004088028875)
\item XXX,XXX. 基于遗传算法的超精密切削加工表面粗糙度预测模型的参数辨识以及切削参数优化[J]. 机械工程学报,2005,41(11):158-162.(EI~收录号:2006039650087)
\item XXX,XXX. Discrete Sliding Mode Cintrok with Fuzzy Adaptive Reaching Law on 6-PEES Parallel Robot[C]. Intelligent System Design and Applications, Jinan, 2006: 649-652.(EI~收录号:20073210746529)
\end{publist}

\noindent\textbf{(二)申请及已获得的专利(无专利时此项不必列出)}
\begin{publist}
\item XXX,XXX. 一种温热外敷药制备方案:中国,88105607.3[P]. 1989-07-26.
\end{publist}

\noindent\textbf{(三)参与的科研项目及获奖情况}
\begin{publist}
\item	XXX,XXX. XX~气体静压轴承技术研究, XX~省自然科学基金项目.课题编号:XXXX.
\item XXX,XXX. XX~静载下预应力混凝土房屋结构设计统一理论. 黑江省科学技术二等奖, 2007.
\end{publist}
%\vfill
%\hangafter=1\hangindent=2em\noindent
%\setlength{\parindent}{2em}
\end{publication}
    % 所发文章
% \begin{ceindex}
  %如果想要手动加索引,注释掉以下这一样,用wordlist环境
\printsubindex*
\end{ceindex}
    % 索引, 根据自己的情况添加或者不添加,选择自动添加或者手工添加。
% \authorization %授权
% %\authorization[scan.pdf] %添加扫描页的命令,与上互斥
% % !Mode:: "TeX:UTF-8"
\begin{acknowledgements}
衷心感谢导师~XXX~教授对本人的精心指导。他的言传身教将使我终生受益。

……

感谢哈工大\LaTeX\ 论文模板\hithesis\ !

\end{acknowledgements}
 %致谢
% % !Mode:: "TeX:UTF-8" 

\begin{resume}
XXXX~年~XX~月~XX~日出生于~XXXX。

XXXX~年~XX~月考入~XX~大学~XX~院(系)XX~专业,XXXX~年~XX~月本科毕业并获得~XX~学学士学位。

XXXX~年~XX~月------XXXX~年~XX~月在~XX~大学~XX~院(系)XX~学科学习并获得~XX~学硕士学位。

XXXX~年~XX~月------XXXX~年~XX~月在~XX~大学~XX~院(系)XX~学科学习并获得~XX~学博士学位。

获奖情况:如获三好学生、优秀团干部、X~奖学金等(不含科研学术获奖)。

工作经历:

\textbf{( 除全日制硕士生以外,其余学生均应增列此项。个人简历一般应包含教育经历和工作经历。)}
\end{resume}
          % 博士学位论文有个人简介
%%%%%%%%%%%%%%%%%%%%%%%%%%%%%%%%%%%%%%%%%%%%%%%%%%%%%%%%%%%%%%%%%%%%%%%%%%%%%%%%
% 博后书序
%%%%%%%%%%%%%%%%%%%%%%%%%%%%%%%%%%%%%%%%%%%%%%%%%%%%%%%%%%%%%%%%%%%%%%%%%%%%%%%%
% \bibliography{reference} % 参考文献
% % !Mode:: "TeX:UTF-8"
\begin{acknowledgements}
衷心感谢导师~XXX~教授对本人的精心指导。他的言传身教将使我终生受益。

……

感谢哈工大\LaTeX\ 论文模板\hithesis\ !

\end{acknowledgements}
 %致谢
% % !Mode:: "TeX:UTF-8" 

\begin{doctorpublication}
\noindent\textbf{(一)发表的学术论文}
\begin{publist}
\item	XXX,XXX. Static Oxidation Model of Al-Mg/C Dissipation Thermal Protection Materials[J]. Rare Metal Materials and Engineering, 2010, 39(Suppl. 1): 520-524.(SCI~收录,IDS号为~669JS,IF=0.16)
\item XXX,XXX. 精密超声振动切削单晶铜的计算机仿真研究[J]. 系统仿真学报,2007,19(4):738-741,753.(EI~收录号:20071310514841)
\item XXX,XXX. 局部多孔质气体静压轴向轴承静态特性的数值求解[J]. 摩擦学学报,2007(1):68-72.(EI~收录号:20071510544816)
\item XXX,XXX. 硬脆光学晶体材料超精密切削理论研究综述[J]. 机械工程学报,2003,39(8):15-22.(EI~收录号:2004088028875)
\item XXX,XXX. 基于遗传算法的超精密切削加工表面粗糙度预测模型的参数辨识以及切削参数优化[J]. 机械工程学报,2005,41(11):158-162.(EI~收录号:2006039650087)
\item XXX,XXX. Discrete Sliding Mode Cintrok with Fuzzy Adaptive Reaching Law on 6-PEES Parallel Robot[C]. Intelligent System Design and Applications, Jinan, 2006: 649-652.(EI~收录号:20073210746529)
\end{publist}

\noindent\textbf{(二)申请及已获得的专利(无专利时此项不必列出)}
\begin{publist}
\item XXX,XXX. 一种温热外敷药制备方案:中国,88105607.3[P]. 1989-07-26.
\end{publist}

\noindent\textbf{(三)参与的科研项目及获奖情况}
\begin{publist}
\item	XXX,XXX. XX~气体静压轴承技术研究, XX~省自然科学基金项目.课题编号:XXXX.
\item XXX,XXX. XX~静载下预应力混凝土房屋结构设计统一理论. 黑江省科学技术二等奖, 2007.
\end{publist}
%\vfill
%\hangafter=1\hangindent=2em\noindent
%\setlength{\parindent}{2em}
\end{doctorpublication}
    % 所发文章
% % !Mode:: "TeX:UTF-8" 
\begin{publication}
\noindent\textbf{发表的相关论文}
\begin{publist}
\item	XXX,XXX. Static Oxidation Model of Al-Mg/C Dissipation Thermal Protection Materials[J]. Rare Metal Materials and Engineering, 2010, 39(Suppl. 1): 520-524.(SCI~收录,IDS号为~669JS,IF=0.16)
\item XXX,XXX. 精密超声振动切削单晶铜的计算机仿真研究[J]. 系统仿真学报,2007,19(4):738-741,753.(EI~收录号:20071310514841)
\item XXX,XXX. 局部多孔质气体静压轴向轴承静态特性的数值求解[J]. 摩擦学学报,2007(1):68-72.(EI~收录号:20071510544816)
\item XXX,XXX. 硬脆光学晶体材料超精密切削理论研究综述[J]. 机械工程学报,2003,39(8):15-22.(EI~收录号:2004088028875)
\item XXX,XXX. 基于遗传算法的超精密切削加工表面粗糙度预测模型的参数辨识以及切削参数优化[J]. 机械工程学报,2005,41(11):158-162.(EI~收录号:2006039650087)
\item XXX,XXX. Discrete Sliding Mode Cintrok with Fuzzy Adaptive Reaching Law on 6-PEES Parallel Robot[C]. Intelligent System Design and Applications, Jinan, 2006: 649-652.(EI~收录号:20073210746529)
\end{publist}

\noindent\textbf{(二)申请及已获得的专利(无专利时此项不必列出)}
\begin{publist}
\item XXX,XXX. 一种温热外敷药制备方案:中国,88105607.3[P]. 1989-07-26.
\end{publist}

\noindent\textbf{(三)参与的科研项目及获奖情况}
\begin{publist}
\item	XXX,XXX. XX~气体静压轴承技术研究, XX~省自然科学基金项目.课题编号:XXXX.
\item XXX,XXX. XX~静载下预应力混凝土房屋结构设计统一理论. 黑江省科学技术二等奖, 2007.
\end{publist}
%\vfill
%\hangafter=1\hangindent=2em\noindent
%\setlength{\parindent}{2em}
\end{publication}
    % 所发文章
% % !Mode:: "TeX:UTF-8" 

\begin{resume}
XXXX~年~XX~月~XX~日出生于~XXXX。

XXXX~年~XX~月考入~XX~大学~XX~院(系)XX~专业,XXXX~年~XX~月本科毕业并获得~XX~学学士学位。

XXXX~年~XX~月------XXXX~年~XX~月在~XX~大学~XX~院(系)XX~学科学习并获得~XX~学硕士学位。

XXXX~年~XX~月------XXXX~年~XX~月在~XX~大学~XX~院(系)XX~学科学习并获得~XX~学博士学位。

获奖情况:如获三好学生、优秀团干部、X~奖学金等(不含科研学术获奖)。

工作经历:

\textbf{( 除全日制硕士生以外,其余学生均应增列此项。个人简历一般应包含教育经历和工作经历。)}
\end{resume}
          % 博士学位论文有个人简介
% % !Mode:: "TeX:UTF-8"
\begin{correspondingaddr}
  \heiti\xiaosi
  \noindent 永久通讯地址: \par
  \noindent email: \par
  \noindent 电话: \par
\end{correspondingaddr}
 %通信地址
%%%%%%%%%%%%%%%%%%%%%%%%%%%%%%%%%%%%%%%%%%%%%%%%%%%%%%%%%%%%%%%%%%%%%%%%%%%%%%%%
\end{document}
% Local Variables:
% TeX-engine: xetex
% End:
