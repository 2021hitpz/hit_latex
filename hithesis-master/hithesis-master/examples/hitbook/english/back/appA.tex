% -*-coding: utf-8 -*-
%%%%%%%%%%%%%%%%%%%%%%%%%%%%%%%%%%%%%%%%%%%%%%%%%%%%%%%%%
\chapter{带章节的附录}[Full Appendix]%
完整的附录内容,包含章节,公式,图表等

%%%%%%%%%%%%%%%%%%%%%%%%%%%%%%%%%%%%%%%%%%%%%%%%%%%%%%%%%
\section{附录节的内容}[Section in Appendix]
这是附录的节的内容

附录中图的示例:
\begin{figure}[htbp]
\centering
\includegraphics[width = 0.4\textwidth]{golfer}
%\bicaption[golfer5]{}{\xiaosi[0]打高尔夫球的人}{Fig.$\!$}{The person playing golf}\vspace{-1em}
\caption{\xiaosi[0]打高尔夫球的人}
\end{figure}

附录中公式的示例:
\begin{align}
a & = b \times c \\
E & = m c^2
\label{eq}
\end{align}

\chapter{这个星球上最好的免费Linux软件列表}[List of the Best Linux Software in our Planet]
\section{系统}

\href{http://fvwm.org/}{FVWM 自从上世纪诞生以来,此星球最强大的窗口管理器。}
推荐基于FVWM的桌面设计hifvwm:\href{https://github.com/dustincys/hifvwm}{https://github.com/dustincys/hifvwm}。

\subsection{hifvwm的优点}

\begin{enumerate}
	\item 即使打开上百个窗口也不会“蒙圈”。计算机性能越来越强大,窗口任务的管理必须要升级到打怪兽级别。
	\item 自动同步Bing搜索主页的壁纸。每次电脑开机,午夜零点自动更新,用户
		也可以手动更新,从此审美再也不疲劳。
	\item 切换窗口自动聚焦到最上面的窗口。使用键盘快捷键切换窗口时候,减少
		操作过程,自动聚焦到目标窗口。这一特性是虚拟窗口必须的人性化设
		计。
	\item 类似window右下角的功能的最小化窗口来显示桌面的功能此处类似
		win7/win10,实现在一个桌面之内操作多个任务。
	\item 任务栏结合标题栏。采用任务栏和标题栏结合,节省空间。
	\item 同类窗口切换。可以在同类窗口之内类似alt-tab的方式切换。
	\item ……
\end{enumerate}

\section{其他}

\href{https://github.com/goldendict/goldendict}{goldendict 星球最强大的桌面字典。}

\href{https://github.com/yarrick/iodine}{iodine,“HIT-WLAN + 锐捷”时代的福音。}

\href{http://www.aircrack-ng.org/}{aircrack,Wifi“安全性评估”工具。}

\href{https://www.ledger-cli.org/}{ledger,前“金融区块链”时代最好的复式记账系统。}

\href{https://orgmode.org/}{orgmode,最强大的笔记系统,从来没有之一。}

\href{https://www.jianguoyun.com/}{坚果云,国内一款支持WebDav的云盘系统,国内真正的云盘没有之一。}

\href{http://www.mutt.org/}{mutt, ``All mail clients suck. This one just sucks less.''}

\section{vim}
实现中英文每一句一行,以及实现每一句折叠断行的简单正则式,tex源码更加乖乖。
\begin{lstlisting}
vnoremap <leader>fae J:s/[.!?]\zs\s\+/\="\r".matchstr(getline('.'), '^\s*')/g<CR>
vnoremap <leader>fac J:s/[。!?]/\=submatch(0)."\n".matchstr(getline('.'), '^\s*')/g<CR>
vnoremap <leader>fle :!fmt -80 -s<CR>
\end{lstlisting}
